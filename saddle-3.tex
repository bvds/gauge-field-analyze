%\documentclass[twocolumn,eqsecnum,aps,]{revtex4-2}
\documentclass[preprint,aps,prd]{revtex4-2}
%\documentclass[eqsecnum,aps]{revtex}
%
%
%  Using Zotero to generate physics.bib file.
%
%  A number of sites set language=en which Zotero
%  passes on to BibTex, but LaTeX package babel doesn't know ``en.''
%  In Zotero, either remove the language field or set it to ``english''
%  before exporting to BibTex.
%\usepackage[english]{babel}
%
%
%  Ubuntu doesn't have the latest RevTex.
%  Download and install using:
%    sudo unzip revtex4-2-tds.zip -d /usr/share/texlive/texmf-dist/
%    sudo mktexlsr /usr/share/texlive/texmf-dist/
%
\bibliographystyle{apsrev4-2}
%
%
\usepackage{amsmath}
\usepackage{amsfonts}
\usepackage{graphicx}
%
\newcommand{\da}{\dagger}  % symbol for Hermitian conjugate dagger
\newcommand{\be}{\begin{equation}}
\newcommand{\eq}{\end{equation}}
\newcommand{\integer}{\mathbb{Z}}       % set of integers
\newcommand{\plaquette}{\Box}
\newcommand{\config}{\mathcal{U}}
\DeclareMathOperator{\SU}{SU}
\DeclareMathOperator{\Tr}{Tr}


\begin{document}
%\draft
\title{Saddle points of the Yang Mills action in 2+1 dimensions}


\author{Brett van de Sande}
%\affiliation{}
\noaffiliation

\begin{abstract}
  We find saddle points of $\SU(N)$ action in three space-time dimensions
  and see characteristics consistent with Witten's Master Field.
\end{abstract}

\pacs{Valid PACS appear here.
{\tt$\backslash$\string pacs\{\}} should always be input,
even if empty.}
\maketitle

%%%%%%%%%%%%%%%%%%%%%%%%%%%%%%%%%%%%%%%%%%%%%%%%%%%%%%%%%%%%%%%%%%%%%%
%%%%%%%%%%%%%%%%%%%%%%%%%%%%%%%%%%%%%%%%%%%%%%%%%%%%%%%%%%%%%%%%%%%%%%
%%%%%%%%%%%%%%%%%%%%%%%%%%%%%%%%%%%%%%%%%%%%%%%%%%%%%%%%%%%%%%%%%%%%%%
\baselineskip .2in

\section{Introduction}
\label{intro}

Since Wilson introduced his lattice action in 1974~\cite{wilson_confinement_1974},
lattice gauge theory has become the tool of choice for
non-perturbative studies of QCD.  Although numerical
calculations using Wilson's lattice have generated numerous
physical results, the nature of the QCD vacuum itself
has remained somewhat of a mystery.

In 1982 't Hooft suggested that center vortices could
explain confinement and subsequent numerical experiments
using maximal abelian gauge have lent some evidence for this
picture.  However, there has been little progress in taking
this picture and creating a full description of the
QCD vacuum state.  [Hugo Reinhardt's ugly paper is a
  heroic attempt in this direction.]

In 1979, Witten pointed out that QCD has some rather remarkable
properties in the $N \to \infty$ limit.  In particular,
expectation values of observables become dominated by a single
configuration of the gauge field, the
``master field''~\cite{witten_1/n_1980}.
Although this idea was compelling, there seemed to be no
way to figure out exactly what this Master Field could be.

To motivate our approach to studying the QCD vacuum, note the following:

\begin{enumerate}

\item Quarks are not important.  In the language of Lattice QCD,
  one says that the quenched approximation (no dynamical
 quarks) is a good approximation to the full theory.

\item The $N\to\infty$ limit of QCD is a good approximation to the
  physical case $N=3$.  This has been demonstrated in numerous
  perturbative and non-perturbative calculations~\cite{lucini_sun_2013}.
  In particular, Teper has shown this to be the case for
  the glueball spectrum in both 2+1~\cite{teper_$mathrmsun$_1998,lucini_$mathrmsun$_2002}
  and 3+1 dimensions~\cite{teper_large-n_2005}.

\item The behavior of $\SU(N)$ gauge theory in 3 space-time dimensions
  is remarkabley similar to the 4 dimensional theory~\cite{teper_$mathrmsun$_1998,lucini_$mathrmsun$_2002,teper_large-n_2005}.
  Although the scale is generated quite differently in each case ---
  in 4 dimensions, it is generated dynamically through quantum
  fluctuations while in 3 dimensions the coupling is dimensionful ---
  once a scale is generated, the theories behave in a very similar manner.

\item In Witten's Master Field picture, the master field is
  a semi-classical field.  That is, it is a saddle point
  of the QCD action.

\end{enumerate}

These facts suggest a research program for studying the QCD vacuum state:
one should look at saddle-points of the gauge field action
in 3 spacetime dimensions, looking for a description that emerges in
the $N\to \infty$ limit.  Hopefully, this description
can then be generalized to the 4 dimensional theory.

I pursue the first step of this program in the following.
We will start with conventional 2+1 dimension lattice gauge
theory field configurations for various values of $N$ and
calculate the nearest associated saddle-point configurations.
These saddle-point field configurations should be
a good starting-point for understanding the 2+1 dimension
vacuum in the $N\to\infty$ limit.

\section{The Wilson action}

The purpose of the section is to set some notation.
We introduce a $D$-dimensional spacetime lattice with lattice
spacing $a$ and sites labeled by $x$.  Define $U_\mu(x) \in \SU(N)$
to be the link going from site $x$ to site $x+a \hat{\mu}$.
If the lattice has $L$ sites in each dimension, the lattice has
$n_L=L^D D \left(N^2-1\right)$ degrees of freedom.
Let lattice configuration
$\config=\left\{U_1(x_1),\ldots\right\}$
be a particular set of link fields that cover the entire lattice.

The associated Wilson action is
%
\be
S = \beta \sum_{x,\, \mu<\nu} \left(1-\frac{1}{N} \Re\, \plaquette_{\mu,\nu}(x)\right) \label{action}
\eq
where
\be
\beta=\frac{2 N}{a g^2}
\eq
so that $S$ becomes the usual Yang-Mills action in the $a\to 0$ limit.
In $D=3$ dimensions, the coupling $g$ has dimensions of
$\left(\mbox{mass}\right)^{1/2}$, consistent with $\beta$ being dimensionless.
Also, $\plaquette_{\mu,\nu}(x)$ is a plaquette lying in the $\mu\nu$-plane:
\be
\plaquette_{\mu,\nu}(x) = \Tr U_\mu(x) U_\nu(x+a \hat{\mu})
U_\mu^\da(x+a\hat{\nu}) U_\nu^\da(x) \; .
\eq
It is sometimes convenient to use the average plaquette,
\be
      \plaquette_\mathrm{avg} = \frac{2}{L^D D (D-1) N}
             \sum_{x,\,\mu<\nu} \Re\,\plaquette_{\mu,\nu}(x) 
\eq
with normalization such that $\plaquette_\mathrm{avg}=1$
for the bare vacuum configuration $U_\mu(x) = \mathbb{I}$.

The Wilson action is invariant under $\SU(N)$ gauge transforms:
%
\be
    U_\mu(x) \to U_G(x) U_\mu(x) U_G^\da(x+a \hat{\mu})
         \; , \quad U_G(x) \in \SU(N) \; .
\eq
%
The lattice has a total of $n_G = L^D\left(N^2-1\right)$ gauge symmetries.

\section{Shifts of the gauge fields}

\begin{figure}
  \[
  \begin{array}{l}
    \mbox{} \hspace{0.2in} x \hspace{0.3in} \sqrt{U_\mu(x)}
    \hspace{0.25in} e^{i w_{a}(\mu, x) T_a}
    \hspace{0.25in} \sqrt{U_\mu(x)} \hspace{0.3in} x+a\hat{\mu} \\[-0.3in]
    %
    %  Graphic generated in Matheatica file ``gauge.nb''
    %
  \includegraphics{link}
  \end{array}
  \]
  \caption{One can think of a shift as applying a color rotation
    to the middle of a lattice link. \label{shift}}
\end{figure}

In order to find a saddle point, it is convenient to modify
the link fields $U_\mu(x)$ in a manner that maintains lattice symmetries.
We define a ``shift'' for each link as
\be
  U_\mu(x) \to \sqrt{U_\mu(x)}\, e^{i w_{a}(\mu, x) T_a}\, \sqrt{U_\mu(x)} \; ,
    \label{shifts}
\eq
where $T_a$ are generators of $\SU(N)$, with normalization
$\delta_{a,b} = 2 \Tr(T_a T_b)$ and the square root is defined
in Appendix~\ref{roots}.
We can then take partial derivatives of $N S/\beta$ with
respect to the $n_L$ shifts $\mathbf{w} = \left(w_a(\mu, x), \ldots\right)$
to obtain the gradient vector $\mathbf{q}$ and Hessian matrix
$H$.  The gradient and Hessian can be used to estimate the
position of the nearest saddle point.

Let us consider the gradient for the link $U_\mu(x)$.
The gradient will involve the $2 (D-1)$ plaquettes that include that link.
Let $F_\mu(x)$ be the associated sum of staples; then
\be
   \frac{N}{\beta} \frac{\partial S}{\partial w_a(\mu, x)} =
   \Im\Tr\left(F_\mu(x) \sqrt{U_\mu(x)} T_a \sqrt{U_\mu(x)}\right) \; .
   \label{grad}
\eq
We can use Eqn.~(\ref{grad}) to construct the gradient vector $\mathbf{q}$.
Likewise, one can calculate the $n_L\times n_L$ Hessian matrix
$H$ from
\be
      \frac{N}{\beta} \frac{\partial^2 S}{\partial w_a(\mu, x)\, \partial w_b(\nu, y)}
\eq
which is nonzero when there is a plaquette which contains both
$U_\mu(x)$ and $U_\nu(y)$.
The full calculation of $H$ is rather lengthy and we will not bore the reader
with the details.

Some shifts $\mathbf{w}$ correspond to gauge transforms.
As we shall see, such shifts are problematic for any saddle
point search, since the action is constant in those directions.
To address this issue, we determine the set of shifts that are
equivalent to infinitesimal gauge transforms.  Consider a gauge transform
at site $x$; its action on link $U_\mu(x)$ is:
\be
U_\mu(x) \to e^{i C_a(x) T_a} U_\mu(x) = U_\mu(x) + i C_a(x) T_a U_\mu(x) +
       \mathrm{O}\!\left(C_a^2(x)\right)
\eq
But this is equal to some infinitesimal shift $w_a(\mu, x)$ on
the same link $U_\mu(x)$:
\be
U_\mu(x) \to U_\mu(x) + i w_a \sqrt{U_\mu(x)}T_a \sqrt{U_\mu(x)} +
       \mathrm{O}\!\left(w_a^2\right)
\eq
%
Equating the two and taking the trace:
\be
w_a = 2 C_b(x) \Tr\left(\sqrt{U_\mu(x)} T_a
                     \sqrt{U_\mu^\da(x)} T_b\right) \; . \label{gs}
\eq
Using Eqn.~(\ref{gs}), one can construct an $n_G \times n_L$ matrix
$G$ that relates shifts to infinitesimal gauge transforms
at each lattice site.
Since the action is invariant under gauge transforms, $G \mathbf{q} = 0$.
However, since the Hessian represents the quadratic term, $G H G^T \neq 0$.

Since we are using the Hessian, we are approximating
the action $S$ as being quadratic in $\mathbf{w}$.  If $\mathbf{w}$
is too large, the quadratic approximation is no longer valid.
Thus, we need some measure of the size of $\mathbf{w}$.
We define two norms:
A lattice-wide $\infty$-norm, taking the ordinary norm at each link:
\be
\left\lVert \mathbf{w}\right\rVert_{\mathrm{max}} =
     \max_{x,\mu} \sqrt{\sum_a w_a^2(\mu, x)}
\eq
as well as an ordinary Euclidean norm, averaged over links
\be
\left\lVert \mathbf{w}\right\rVert_2 =
     \sqrt{\frac{1}{L^D D} \sum_{x, \mu, a} w_a^2(\mu, x)}
        \; .
\eq



\section{Calculating the saddle point}
\label{saddle}

\begin{figure}
\includegraphics{eigenAll}
\caption{Histogram of the eigenvalues of $H$ for a $6^3$ lattice,
  $\beta = 8.175$, $N=3$.  We see a peak around zero due to the
  $\SU(3)$ gauge symmetry.
  \label{eigenAll}}
\end{figure}

In principle, to find the nearest saddle point, one would
find $H$ and $\mathbf{q}$, solve the linear system
\be
    H \mathbf{y} = \mathbf{q} \label{linear1}
\eq
and shift the gauge fields using $\mathbf{w} = -\mathbf{y}$.
Since the action $S$ is not quadratic in $\mathbf{w}$, one
would then iterate this process to find the actual saddle point.

However, in practice, this procedure does not work.  To see why, let us
look at the eigenvalues of $H$.  As seen in Figure~\ref{eigenAll},
we see that the eigenvalues are strongly peaked around zero,
due to the gauge symmetry of the theory.  Thus, the linear
system (\ref{linear1}) will be nearly singular and
the solution will be numerically unstable.  We must be more careful.

To address this issue, we remove directions in shift-space corresponding
to infinitesimal gauge transforms.  To do this, we start
by calculating the null-space $P_G$ of the gauge transform matrix $G$,
\be
G P_G^T = 0
\eq
where $P_G P_G^T = \mathbb{I}$.  Then we project $H$ and $\mathbf{q}$ onto
the space of non-gauge-transform shifts:
\be
         H^\prime = P_G H P_G^T \quad \mbox{and} \quad
         \mathbf{q}^\prime = P_G \mathbf{q} \; .
\eq
Then we can solve the linear system
\be
   H^\prime \mathbf{y}^\prime = \mathbf{q}^\prime \label{linear2}
\eq
and set $\mathbf{w} = - P_G^T \mathbf{y}^\prime$.  However, this is
still insufficient to yield a well-behaved linear system.

\begin{figure}
\includegraphics{shiftPairs}
\caption{Scatter plot of eigenvalue-gradient pairs of $H^\prime$
  for a $6^3$ lattice, $\beta = 8.175$, $N=3$.  The pairs above the
  v-shaped wedge correspond to large shifts and will dominate
  the solution to the linear system, Eqn.~(\ref{linear2}).
  \label{shiftPairs}}
\end{figure}

To better understand this issue, let us find the eigenvalues
$\left(\lambda_1, \lambda_2, \ldots\right)$ and
eigenvectors $\mathcal{O}=\left(\mathbf{v}_1, \mathbf{v_2}, \ldots\right)$
of $H^\prime$ and express Eqn.~(\ref{linear2}) in the eigenspace of $H^\prime$:
\be
\begin{pmatrix}
    \lambda_1 & & \\
    & \lambda_2 & \\
    & & \ddots  \end{pmatrix} \overline{\mathbf{y}} =
  \overline{\mathbf{q}} \label{linear3}
\eq
where $\overline{\mathbf{y}} = \mathcal{O} \mathbf{y}^\prime$ and
$\overline{\mathbf{q}}  = \mathcal{O} \mathbf{q}^\prime$.
The solution is simply
\be
    \overline{y}_i = \frac{\overline{q}_i}{\lambda_i} \; .
\eq
In any case where $\left|\lambda_i\right|\ll
\left|\overline{q}_i\right|$,
there will be a large shift that will dominate the solution
to the linear system.
Thus, we impose a cutoff $\Lambda_2$:
\be
    \frac{\left|\overline{q}_i\right|\left\lVert P_G^T \mathbf{v}_i\right\rVert_2}{\left|\lambda_i\right|}
     < \Lambda_2 \label{lambda2}
\eq
%
and set $\overline{y}_i=0$ whenever the cutoff is violated.
That is, we remove any eigenpairs inside the v-shaped
region shown in Fig.~\ref{shiftPairs}.  For additional safety, we
we also impose maximum value cutoff
\be
    \frac{\left|\overline{q}_i\right|
      \left\lVert P_G^T \mathbf{v}_i\right\rVert_\mathrm{max}}
    {\left|\lambda_i\right|}
    < \Lambda_\mathrm{max} \; .
\eq
Thus,
\be
      \mathbf{y} = P_g^T \left. \mathcal{O}^T
                    \overline{\mathbf{y}}\right|_{\Lambda_2,\Lambda_\mathrm{max}} \; .
\eq

Finally, since the gauge symmetry of the action is non-linear,
we do not want to shift the fields by too much before recalculating
the matrix $G$.  We want to ensure the shift is small
enough on each link to be consistant with the quadratic expansion.
Thus, it is useful to add a damping factor $\gamma$ and
further rescale by $\Lambda_s$, when needed:
\be
    \mathbf{w} = - \gamma \mathbf{y} \min\left(1, \frac{\Lambda_s}{\left\lVert \mathbf{y}\right\rVert_\mathrm{max}}\right) \; .
\eq
Typically, $\Lambda_s = 1$ or $2$ is used.

Armed with these three tools:  removing infinitesimal-gauge-transform shifts,
imposing the $\Lambda$ cutoffs, and using the damping factor $\gamma$,
we can apply the shift iteratively and find saddle-point configurations
that are of physical interest.

\section{Trajectories}

\begin{figure}
\includegraphics{trajectory3}
\caption{Several trajectories are shown for a $16^3$ lattice configuration,
  $N=3$, $\beta=28$.
  The starting configuration $\config_0$ was generated using standard Euclidean
  Lattice Monte Carlo;
  {\bf 2b} was generated with $\gamma=0.1$, $\Lambda_\mathrm{max} = 2$,
  and $\Lambda_2=0.04434$;
  {\bf 2c} was generated with $\gamma=0.5$, $\Lambda_\mathrm{max} = 2$,
  and $\Lambda_2=0.02217$;
  {\bf 2d} was generated with $\gamma=0.1$, $\Lambda_\mathrm{max} = 2$,
  and $\Lambda_2=0.02217$.
  For {\bf lc}, $\gamma=0.1$, off-diagonal blocks of $H$ were dropped,
  and a shift
  was applied to a link when $\left\lVert\mathbf{w}\right\rVert<0.327$.
  \label{trajectory3}}
\end{figure}
  
We will apply our procedure to $\SU(3)$, $16^3$ and $20^3$ lattice
configurations, as well as $\SU(4)$, $16^3$ lattice
configurations.  Details of the numerical calculation
are discussed in Appendices~\ref{krylov} and \ref{configurations}.

By applying the procedure discussed in Section~\ref{saddle} iteratively,
we produce a trajectory of lattice configurations $\config_0, \config_1, \ldots$
in configuration space.  In Fig.~\ref{trajectory3}, we show some trajectories
in terms of the string tension $\sigma a^2$ and $\plaquette_\mathrm{avg}$.

Lucini and Teper report a string tension $\sigma a^2=0.016282(76)$
for $N=3$, and $\beta=28$ (Ref.~\cite{lucini_$mathrmsun$_2002}, Table~5).
For the initial configuration $\config_0$ used in Fig.\ref{trajectory3},
we measure $\sigma a^2=0.0155(56)$;
with errors that remain about that size for $\config_k$, $k>0$.
For $\config_0$, $\plaquette_\mathrm{avg}=0.902$ and the
action is dominated by perturbative fluctuations of the gauge fields.
One can use the string tension to
estimate $\plaquette_\mathrm{avg}$ where we expect the action to become
dominated by the string-like behavior of the theory:
\be
\plaquette_\mathrm{avg} \gtrsim  1 - \sigma a^2 = 0.984 \; .
\eq


Ideally, $\plaquette_\mathrm{avg}$ will increase steadily and the string tension will remain relatively constant as we apply our iterative procedure.
This is illustrated by the trajectory {\bf 2d}.
In cases where $\gamma$ or $\Lambda_2$ is too large, we find that the
string tension does not remain stable, as illustrated by
trajectories {\bf 2b} and {\bf 2c}.

We also have investigated an ad-hoc procedure where
the off-diagonal color blocks of $H$ are dropped (that is, the
saddle point search is applied to each link individually).
This procedure is much easier to apply numerically.
As shown in trajectory {\bf lc},
we see that that the string tension can remain relatively stable,
but $\config_k$ eventually degrades to the trivial vacuum.

{\bf  To do:
  finish $\SU(3)$, $L^3=20^3$ calculation and
  add results for $\SU(4)$}


\section{Spectrum}

\begin{figure}
   configuration $\config_0$\\
   \includegraphics{corrOriginal}\\
   configuration $\config_{10}$\\
   \includegraphics{corr2d10}\\
   configuration $\config_{20}$\\
   \includegraphics{corr2d20Even}
\caption{Correlations of the even plaquette operator for several points
  along trajectory {\bf 2d} shown in Fig.~\ref{trajectory3}.
  Also shown is an exponential corresponding to Teper's value
  for the lowest $0^{++}$ glueball.
  \label{corr2dEven}}
\end{figure}

\begin{figure}
   \includegraphics{corr2d20Odd}
   \caption{Correlations of the odd plaquette operator for
     lattice configuration
     $\config_{20}$ on trajectory {\bf 2d} shown in Fig.~\ref{trajectory3}.
  Also shown is an exponential corresponding to Teper's value
  for the lowest $0^{--}$ glueball.
  \label{corr2dOdd}}
\end{figure}

\begin{figure}
   \includegraphics{corrLink995}
   \caption{Correlations of the even plaquette operator for
     the $\plaquette_\mathrm{avg} \approx 0.995$ lattice configuration
     on trajectory {\bf lc} shown in Fig.~\ref{trajectory3}.
    Also shown is an exponential corresponding to Teper's value
     for the lowest $0^{++}$ glueball.
     We do not see the desired exponential behavior.
  \label{corrLink995}}
\end{figure}

We can look at other properties of the gauge configuration
as we move along the trajectory.  In particular, let us look
at the correlators of the charge conjugation even plaquette operator
\be
  \left\langle
  \left(\Re \plaquette_{\mu,\nu}(x) - \plaquette_\mathrm{avg}\right)
  \left(\Re \plaquette_{\alpha,\beta}(y) - \plaquette_\mathrm{avg}\right)
  \right\rangle
\eq
as a function of the separation of the two plaquettes
\be
\mathrm{separation} =
        \left\lVert \left(x+a \hat{\mu}/2+ a\hat{\nu}/2\right) -
             \left(y+a \hat{\alpha}/2+ a\hat{\beta}/2\right)
             \right\rVert \; .
\eq
This correlator couples to the $0^{++}$ glueball states.

In Fig.~\ref{corr2dEven}, we plot the correlation for various relative
orientations of the two plaquettes for several points along the
best trajectory in~\ref{trajectory3}.  We also plot an exponential
corresponding to Teper's value for the $0^{++}$ glueball,
mass $0.5517(38)/a$ for $\beta=28$, $L^3=23^3$~\cite{teper_$mathrmsun$_1998}.
The exponential only demonstrates that there is qualitative agreement.
Since the correlator also has contributions from the excited $0^{++}$
states, a quantitative fit would require a more extensive analysis,
with additional operators in the correlation function.  However, one
would expect the single plaquette operator to couple very strongly
to the lowest state and not so strongly to the excited states.

Likewise, we can look at the charge conjugation odd correlator
\be
  \left\langle
  \left(\Im \plaquette_{\mu,\nu}(x) - \plaquette_\mathrm{avg}\right)
  \left(\Im \plaquette_{\alpha,\beta}(y) - \plaquette_\mathrm{avg}\right)
  \right\rangle   \; ,
\eq
corresponding to the $0^{--}$ glueball states; see Fig.~\ref{corr2dOdd}.
Also shown is an exponential corresponding to the lowest $0^{--}$ state,
mass $0.8133(57)/a$ \cite{teper_$mathrmsun$_1998}.

In the case where the off-diagonal color blocks of $H$ are dropped,
we do not see the desired exponential behavior in the correlator,
Fig.~\ref{corrLink995}.

\section{Gauge Fields}

{\bf To do:  look at the fields themselves.

  Ideas:
  \begin{itemize}
  \item  Look at $A_{\mu,a}$ directly in the $A_0=0$ (axial) gauge.

  \item Find some nice way to express the Faraday tensor $F_{\mu,\nu}$ and
    plot it.

  \item Impose fixed boundary conditions on a cylinder where we fix the
    link fields to correspond to one center charge inside the
    cylinder.
    
  \end{itemize}
}


\section{Conclusions}

The master field idea is a statement about the $N\to \infty$ limit
of $\SU(N)$ gauge theory.  However, if finite $N$ is similar
to the $N\to\infty$ limit, we should expect to properties of
the master field emerge, approximately, at finite $N$.
As an example of this, we
see that a saddle point of a single lattice configuration
contains the necessary physics, at least approximately, to produce the
low-energy spectrum of the theory.

\vspace{10mm}
\noindent {Acknowledgments}:

\appendix

\section{Roots of $\SU(N)$ matrices}
\label{roots}

In general, for matrix $U_c$ belonging to a connected Lie group,
the matrix power $U_c^t$, $t\in[0,1]$ should be a
smooth map from $[0,1]$ onto the group manifold where
$U_c^0=\mathbb{I}$, $U_c^1=U_c$, and $U_c^s U_c^t = U_c^{s+t}$;
the image of the map should have minimal length.
In general, a unitary matrix $U$ can be factored as
%
\be
U = V^\da \begin{pmatrix}
    \mathrm{e}^{i \lambda_1} & & &\\
    & \mathrm{e}^{i \lambda_2} & &\\
    & & \ddots & \\
    & & & \mathrm{e}^{i \lambda_N}\end{pmatrix} V
\eq
%
where $V$ is a unitary matrix and $-\pi < \lambda_i < \pi$.
In the special case $U \in \SU(N)$, the $\det(U)=1$ condition implies that
\be
\sum_i \lambda_i = 2 \pi m\;, \quad m\in\integer \;.
\eq
When taking matrix roots, cases where $m\neq 0$ present a
difficulty.  However, one can always shift $\lambda_i$ by
multiples of $2\pi$:
\be
\lambda_i \to \lambda_i^\prime = \lambda_i + 2 \pi n_i\;,\quad
n_i\in\integer \; ,
\eq
such that $\sum_i \lambda_i^\prime = 0$ and
$\lambda_i^\prime - \lambda_j^\prime < 2 \pi$.
%
\begin{figure}
\includegraphics{hex3}
  \caption{In the case of $\SU(3)$, $(\lambda_1^\prime,\lambda_2^\prime)$ lie in the shaded hexagonal region.
    Elements of the center of the group $z$, $z^2$, lie on the vertices of the hexagon. \label{hexagon}}
\end{figure}
%
The $\SU(3)$ case is shown in Fig.~\ref{hexagon}.
Using $\lambda_i^\prime$, we define $U^t$ as:
\be
U^t = V^\da \begin{pmatrix}
    \mathrm{e}^{i\lambda_1^\prime t} & & &\\
    & \mathrm{e}^{i\lambda_2^\prime t} & &\\
    & & \ddots & \\
    & & & \mathrm{e}^{i\lambda_N^\prime t}\end{pmatrix} V \; .
\eq
This definition of $U^t$ fulfills the conditions listed above.
Similarly, we define the logarithm of $U$ as:
\be
\log U = i V^\da \begin{pmatrix}
    \lambda_1^\prime & & &\\
    & \lambda_2^\prime & &\\
    & & \ddots & \\
    & & & \lambda_N^\prime\end{pmatrix} V
\eq

\section{Solving a constrained linear system}
\label{krylov}

For lattices of interest, the basis size $n_L$ is large enough
that direct solution is not possible and one must resort to Krylov
space methods (sparse matrix methods).

The Hessian matrix $H$ is sparse, with an $(N^2-1)\times(N^2-1)$
color block structure and $6D+1$ nonzero blocks per row/column.
The gauge transform matrix $G$ has the same color block structure
with $2 D$ nonzero blocks per row and $2$ nonzero blocks per column.

In order to preserve the sparsity of the matrices, we never
explicitly construct the space of non-transform shifts used in
Eqn.~(\ref{linear2}).  Instead, we apply the projection
operator $P_G$ to any Krylov space vector $\mathbf{v}$ using:
\be
    P_G \mathbf{v} = \mathbf{v} - G^T \mathbf{u} \; ,
\eq
where $\mathbf{u}$ is a solution of the linear system
\be
  G G^T \mathbf{u} = G \mathbf{v} \; .\label{gcg}
\eq
Eqn.~(\ref{gcg}) is solved using the conjugate gradient
algorithm, subroutine MINRES-QLP~\cite{choi_algorithm_2014}.
The matrix $G G^T$ is well-conditioned, so this typically converges
in less than 40 steps.

We use the Lanczos library nu-TRLan~\cite{yamazaki_adaptive_2010},
modified so that $P_G$ is applied to any Krylov space vector, to
find the smallest eigenpairs of $\left(P_g H\right)^2$.  We then find
$M$ eigenvectors $\left\{\mathbf{r}_1, \mathbf{r}_2, \ldots\right\}$
associated with eigenpairs $\left(\lambda_i^2, \mathbf{r}_i\right)$
that violate the $\Lambda$ cutoffs,
\begin{eqnarray}
   \frac{ \mathbf{r}_i \mathbf{\cdot} \mathbf{q}
   \left\lVert \mathbf{r}_i\right\rVert_2}{\left|\lambda_i\right| }
     &>& \Lambda_2  \quad\mbox{or} \\
   \frac{ \mathbf{r}_i\mathbf{\cdot} \mathbf{q}
   \left\lVert \mathbf{r}_i\right\rVert_\mathrm{max}}{\left|\lambda_i\right| }
     &>& \Lambda_\mathrm{max} \; .
\end{eqnarray}
To ensure that we have found all the eigenpairs that violate the cutoff,
we may need to calculate as many as 2000 to 4000 of the lowest eigenpairs
in the Lanzos routine.  As one moves along a trajectory towards the
saddle point, the gradient vector decreases in magnitude and fewer eigenpairs
exceed the $\Lambda$ cutoffs.

Then we solve the linear system $H \mathbf{y} = \mathbf{q}$
using the conjugate gradient algorithm,
subroutine MINRES-QLP~\cite{choi_algorithm_2014}, modified so that
the projection operators $P_G$ and 
$\mathbb{I}- \sum_{k=1}^M \mathbf{r}_k \otimes \mathbf{r}_k$
are applied to any Krylov space vector.

\section{Lattice configurations}
\label{configurations}

Lattice configurations were generated using the Chroma
code~\cite{edwards_chroma_2005}.
For $N>3$ it was necessary to add partial pivoting~\cite{golub_matrix_1996}
to the Chroma routine that calculates matrix determinants.  We then
verified that Chroma-generated values of $\plaquette_\mathrm{avg}$ agree with
Teper's results~\cite{lucini_$mathrmsun$_2002} for $N=3,4$.

We used ... {\bf details of Monte Carlo: thermalizing steps,
  over-relaxation, et cetera.
  Details of Polyakov loop calculations.}

\bibliography{physics}  % Use physics.bib

\end{document}
