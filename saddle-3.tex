%\documentclass[twocolumn,eqsecnum,aps]{revtex4}
\documentclass[preprint,aps]{revtex4}
%\documentclass[eqsecnum,aps]{revtex}
%
\usepackage{amsmath}
\usepackage{amsfonts}
\usepackage{graphicx}
%
\newcommand{\da}{\dagger}  % symbol for Hermitian conjugate dagger
\newcommand{\be}{\begin{equation}}
\newcommand{\eq}{\end{equation}}
\newcommand{\integer}{\mathbb{Z}}       % set of integers
\newcommand{\plaquette}{p}
\DeclareMathOperator{\SU}{SU}
\DeclareMathOperator{\Tr}{Tr}


\begin{document}
%\draft
\title{Saddle points of the Yang Mills action in 2+1 dimensions}


\author{Brett van de Sande}
%\affiliation{}


\begin{abstract}
  We find saddle points of $\SU(N)$ action in three space-time dimensions
  and see characteristics consistant with Witten's Master Field.
\end{abstract}

\pacs{Valid PACS appear here.
{\tt$\backslash$\string pacs\{\}} should always be input,
even if empty.}
\maketitle

%%%%%%%%%%%%%%%%%%%%%%%%%%%%%%%%%%%%%%%%%%%%%%%%%%%%%%%%%%%%%%%%%%%%%%
%%%%%%%%%%%%%%%%%%%%%%%%%%%%%%%%%%%%%%%%%%%%%%%%%%%%%%%%%%%%%%%%%%%%%%
%%%%%%%%%%%%%%%%%%%%%%%%%%%%%%%%%%%%%%%%%%%%%%%%%%%%%%%%%%%%%%%%%%%%%%
\baselineskip .2in

\section{Introduction}
\label{intro}

Since Wilson introduced his lattice action in 1976,
lattice gauge theory has become the tool of choice for
non-perturbative studies of QCD.  Although numerical
calculations using Wilson's lattice have generated numerous
physical results, the nature of the QCD vacuum itself
has remained somewhat of a mystery.

In 1982 [?] 't Hooft suggested that center vortices could
explain confinement and subsequent numerical experiments
using maximal abelian gauge have lent some evidence for this
picture.  However, there has been little progress in taking
this picture and creating a full description of the
QCD vacuum state.  [Hugo Reinhardt's ugly paper is a
  heroic attempt in this direction.]

In 1979, Witten pointed out that QCD has some rather remarkable
properties in the $N \to \infty$ limit.  In particular,
expectation values of observables become dominated by a single
configuration of the gauge field, the ``master field.''
Although this idea was compelling, there seemed to be no
way to figure out exactly what this Master Field is.

In studying the QCD vacuum, one should note the following:

\begin{enumerate}

\item Quarks are not important.  In the langauge of Lattice QCD,
  one says that the quenched approximation (no dynamical
 quarks) is a good approximation to the full theory.

\item The $N\to\infty$ limit of QCD is a good approximation to the
  physical case $N=3$.  This has been demonstrated in numerous
  perturbative and non-perturbative calculations [cite review paper].
  In particular, Mike Teper has shown this to be the case for
  the glueball spectrum [cite teper].

\item The behavior of $\SU(N)$ gauge theory in 3 space-time dimensions
  is remarkabley similar to the 4 dimensional theory [cite Teper].
  Although the scale is generated quite differently in each case ---
  in 4 dimensions, it is generated dynamically through quantum
  flucutations while in 3 dimensions the coupling is dimensionful ---
  once a scale is generated, the theories behave in a very similar manner.

\item In Witten's Master Field picture, the master field is
  a semi-classical field.  That is, it is a saddle point
  of the QCD action.

\end{enumerate}

These facts suggest a research program for studying the QCD vacuum state:
one should look at saddle-points of the gauge field action
in 3 spacetime dimensions, looking for a description that emerges in
the $N\to \infty$ limit.  Hopefully, this description
can then be applied to the 4 dimensional theory.

I pursue the first step of this program in the following.
We will start with conventional $D=3$ lattice gauge
theory field configurations for various values of $N$ and
calculate the nearest associated saddle-point configurations.

We believe that these saddle-point field configurations
are a good starting-point for understanding the 2+1 theory
vacuum in the $N\to\infty$ limit.

\section{Saddle points of the Wilson action}
\label{two}

%  Need to finish defining notation
%
We introduce a $D-dimensional$ spacetime lattice with
sites $x$ representing the quark fields (which we will ignore)
and links representing the gauge fields:
\be
U(x,x+\hat{\mu})  = \mathrm{e}^{i A_a T_a}
\eq
where $T_a$ are the generators of $\SU(N)$ and the index
$a\in\{1, \ldots, N^2-1\}$.  The link fields $A_a$ represent
the gauge fields of the theory.

The associated action [cite Wilson] is
%
\be
S = \mathrm{e}^{-\beta \left(1-\sum_\plaquette U_\plaquette\right)}
\eq
Next define gauge symmetry.

Thus, a lattice with $L^3$ sites has $L^3 D \left(N^2-1\right)$
degrees of freedom.


\section{Conclusions}


\vspace{10mm}
\noindent {Acknowledgements}:

\vspace{10mm}
\noindent {\bf APPENDIX: Roots of $\SU(N)$ matrices}

\vspace{10mm}

In general, for some Lie group matrix $U$, the matrix power
$U^t$, $t\in[0,1]$ should be a smooth map from $[0,1]$
onto the group manifold where $U^0=\mathbb{I}$ and $U^1=U$. 
In general, a unitary matrix $U$ can be factored as
%
\be
U = V^\da \begin{pmatrix}
    \mathrm{e}^{i \lambda_1} & & &\\
    & \mathrm{e}^{i \lambda_2} & &\\
    & & \ddots & \\
    & & & \mathrm{e}^{i \lambda_N}\end{pmatrix} V
\eq
%
where $V$ is a unitary matrix and $-\pi < \lambda_i < \pi$.
In the case $U \in \SU(N)$ the $\det(U)=1$ condition implies that
\be
\sum_i \lambda_i = 2 \pi m\;, \quad m\in\integer \;.
\eq
For taking matrix roots, cases where $m\neq 0$ present a
difficulty.  However, one can always shift $\lambda_i$ by
multiples of $2\pi$:
\be
\lambda_i \to \lambda_i^\prime = \lambda_i + 2 \pi n_i\;,\quad
n_i\in\integer
\eq
such that $\sum_i \lambda_i^\prime = 0$ and
$\lambda_i^\prime - \lambda_j^\prime < 2 \pi$.
%
\begin{figure}

\includegraphics[width=2in]{hex3}
  \caption{In the case of $\SU(3)$, $(\lambda_1^\prime,\lambda_2^\prime)$ lie in the shaded hexagonal region.
    Elements of the center of the group $z$, $z^2$, lie on the vertices of the hexagon.}
\end{figure}
%
Using $\lambda_i^\prime$, we define $U^t$ as:
\be
U^t = V^\da \begin{pmatrix}
    \mathrm{e}^{i\lambda_1^\prime t} & & &\\
    & \mathrm{e}^{i\lambda_2^\prime t} & &\\
    & & \ddots & \\
    & & & \mathrm{e}^{i\lambda_N^\prime t}\end{pmatrix} V
\eq
Similarly, we define the logarithm of $U$ as:
\be
\log U = i V^\da \begin{pmatrix}
    \lambda_1^\prime & & &\\
    & \lambda_2^\prime & &\\
    & & \ddots & \\
    & & & \lambda_N^\prime\end{pmatrix} V
\eq

\begin{thebibliography}{99}


\end{thebibliography}

\end{document}
