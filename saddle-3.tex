%\documentclass[twocolumn,eqsecnum,aps,]{revtex4-2}
\documentclass[preprint,aps,prd]{revtex4-2}
%\documentclass[eqsecnum,aps]{revtex}
%
%
%  Using Zotero to generate physics.bib file.
%
%  In Zotero, export as "UTF-8 without BOM" to avoid an error in Ubuntu 20.
%
%  A number of sites set language=en which Zotero
%  passes on to BibTex, but LaTeX package babel doesn't know ``en.''
%  In Zotero, either remove the language field or set it to ``english''
%  before exporting to BibTex.
%\usepackage[english]{babel}
%
%
%  Ubuntu doesn't have the latest RevTex.
%  Download and install using:
%    sudo unzip revtex4-2-tds.zip -d /usr/share/texlive/texmf-dist/
%    sudo mktexlsr /usr/share/texlive/texmf-dist/
%
%  RevTex 4.2 does not properly handle very long author lists.
%  Run this to truncate the author list for the 2014 Brambilla article:
%    perl -pi -e 's/and Eidelman,.*}/and others}/' physics.bib
%
\bibliographystyle{apsrev4-2}
%
\usepackage{amsmath}
\usepackage{mathtools} % For MoveEqnLeft
\usepackage{amsfonts}
\usepackage{graphicx}
\usepackage{afterpage} % For figure placement
%
\newcommand{\da}{\dagger}  % symbol for Hermitian conjugate dagger
\newcommand{\be}{\begin{equation}}
\newcommand{\eq}{\end{equation}}
\newcommand{\integer}{\mathbb{Z}}       % set of integers
\newcommand{\zentrum}{\mathcal{Z}}       % set of integers
\newcommand{\plaquette}{\Box}
\newcommand{\config}{\mathcal{U}}
\newcommand{\orelax}{\xi}
\newcommand{\heigen}{h}
\newcommand\wilson[4]{\Omega_{#1, #2}\left(#3,#4\right)}
\DeclareMathOperator{\SU}{SU}
\DeclareMathOperator{\Tr}{Tr}
\newcommand\fnorm[1]{\left\lVert #1 \right\rVert_\mathrm{F}}
\newcommand\znorm[1]{\left\lVert #1 \right\rVert_\zentrum}
\newcommand\cov[2]{\mathrm{cov}\!\left(#1, #2\right)}
\newcommand{\meyerscale}{r_\mathrm{s}}

\begin{document}
%\draft
\title{Saddle points of the Yang Mills action in 2+1 dimensions}


\author{Brett van de Sande}
%\affiliation{}
\noaffiliation

\begin{abstract}
  We find saddle points of $\SU(N)$ Yang Mills action in three
  space-time dimensions for $N=3,4$.
  Using Landau gauge, we examine various correlators of the gauge fields
  in the vicinity of the saddle-point.
  The ultimate goal of
  this work to to determine the functional form of the vacuum gauge
  field in the large $N$ limit, Witten's Master Field.
\end{abstract}

\pacs{Valid PACS appear here.
{\tt$\backslash$\string pacs\{\}} should always be input,
even if empty.}
\maketitle

%%%%%%%%%%%%%%%%%%%%%%%%%%%%%%%%%%%%%%%%%%%%%%%%%%%%%%%%%%%%%%%%%%%%%%
%%%%%%%%%%%%%%%%%%%%%%%%%%%%%%%%%%%%%%%%%%%%%%%%%%%%%%%%%%%%%%%%%%%%%%
%%%%%%%%%%%%%%%%%%%%%%%%%%%%%%%%%%%%%%%%%%%%%%%%%%%%%%%%%%%%%%%%%%%%%%
\baselineskip .2in

\section{Introduction}
\label{intro}

Since Wilson introduced his lattice action in 1974~\cite{wilson_confinement_1974},
lattice gauge theory has become the tool of choice for
non-perturbative studies of QCD.  Although numerical
calculations using Wilson's lattice have generated numerous
physical results, the nature of the QCD vacuum itself
has remained somewhat of a mystery~\cite{brambilla_qcd_2014}.

% In 1982 't Hooft suggested that center vortices could
% explain confinement and subsequent numerical experiments
% using maximal abelian gauge have lent some evidence for this
% picture.  However, there has been little progress in taking
% this picture and creating a full description of the
% QCD vacuum state.  [Hugo Reinhardt's ugly paper is a
% heroic attempt in this direction.]

In 1979, Witten pointed out that QCD has some rather remarkable
properties in the $N \to \infty$ limit.  In particular,
expectation values of gauge invariant observables become dominated by a single
classical configuration of the gauge field, the
``master field''~\cite{witten_1/n_1980}.
Although this idea was compelling, there seemed to be no
way to figure out exactly what this master field could be.
However, in the intervening 40 years, we have learned a number
of things that make the quest for this master field perhaps
attainable.  We note the following:

\begin{enumerate}

\item Quarks are not important.  In the language of Lattice QCD,
 one says that the quenched approximation (no dynamical
 quarks) is a good approximation to the full theory.  Numerous
 lattice calculations have given us a good understanding
 of the size and nature of this approximation.

\item The $N\to\infty$ limit of QCD is a good approximation to the
 physical case $N=3$.  This has been demonstrated in numerous
 perturbative and non-perturbative calculations~\cite{lucini_sun_2013}.
 In particular, Teper and collaborators have shown this to be the case for
 the glueball spectrum in both 2+1~\cite{teper_$mathrmsun$_1998,lucini_$mathrmsun$_2002,athenodorou_sun_2017}
 and 3+1 dimensions~\cite{teper_large-n_2005}.

\item The behavior of $\SU(N)$ gauge theory in 3 space-time dimensions
  is remarkably similar to the 4 dimensional theory~\cite{teper_$mathrmsun$_1998,lucini_$mathrmsun$_2002,teper_large-n_2005}.
  Although the scale is generated quite differently in each case---in
  4 dimensions, it is generated dynamically through quantum
  fluctuations while in 3 dimensions the coupling is dimensionful---once
  the scale exists, the theories behave in a very similar manner.

\end{enumerate}
%
The idea of exploiting this combination (pure gauge theory, 2+1 dimensons,
and large $N$) to study QCD  is not new.  Indeed, a number of
researchers, most notably Mike Teper and various collaborators,
have pursued this strategy vigorously over the last 25
years~\cite[Section~5.2]{lucini_sun_2013}.
The main focus of that work has been to connect gauge
theory to some effective string theory, with the main result being
that string theory works suprisingly well
as a description of large-$N$ gauge theory~\cite{brandt_effective_2016}.
In particular, the lowest
order string theory (the Nambu-Goto string) describes the confining
potential at length scales well below what one would expect
from dimensional arguments.

The concept of the master field only exists in the $N\to \infty$
limit.  At finite $N$ and finite lattice spacing, we would expect the associated
saddle-point to exist in some approximate sense.  In a numerical
procedure, one might expect that a search for the saddle-point
would initially move toward a physically interesting field
configuration, but eventually move to toward the bare vacuum state
or some other physically uninteresting state.  This was what was seen
in various lattice cooling studies in the 1980s where one
would reach a ``semiclassical plateau'' before eventually
decaying to the bare vacuum state~\cite{campostrini_cooling_1989,teper_cooling_1994}.  As $N$ increases, one would expect this ``semiclassical plateau''
to be come more and more robust.

In the present work, we will turn our attention back
to this ``semi-classical plateau'' and attempt to understand
the associated large $N$ limit.  Rather than locally
minimizing the action, we will look for saddle-points of
the action, calculated globally, using Newton's method.
Our calculations will focus on $N=3,4$.
Also, rather than measuring physical observables, we will focus
on the fields themselves, using Landau gauge.
Landau gauge gives us the best hope
of connecting the Wilson link variables to the continuum
Yang Mills fields.

In Section~\ref{def}, we introduce the Wilson action and
a define a number of useful norms.  In Section~\ref{observables}
{\bf ...  to do ...}

\section{Definitions}
\label{def}

The purpose of the section is to set some notation.

\subsection{The Wilson action}

We introduce a periodic $D$-dimensional spacetime lattice with lattice
spacing $a$ and sites labeled by $x$, with a total of $V_L$ sites.
The lattice has $L_\mu$ sites in direction $\mu$;
in the case of a cubic lattice, we will drop the $\mu$ subscript
and $V_L = L^D$.
Define $U_\mu(x) \in \SU(N)$ to be the link going from site $x$ to
site $x+a \hat{\mu}$.
The lattice has $n_L=V_L D \left(N^2-1\right)$ degrees of freedom.
Let lattice configuration
$\config=\left\{U_1(x_1),\ldots\right\}$
be a particular set of link fields that cover the lattice.

If $\plaquette_{\mu,\nu}(x)$ is a plaquette lying in the $\mu\nu$-plane,
\be
\plaquette_{\mu,\nu}(x) = \Tr U_\mu(x) U_\nu(x+a \hat{\mu})
U_\mu^\da(x+a\hat{\nu}) U_\nu^\da(x) \; ,
\eq
then the Wilson action is
%
\be
S = \beta \sum_{x,\, \mu<\nu} \left(1-\frac{1}{N} \Re\, \plaquette_{\mu,\nu}(x)\right) \label{action}
\eq
where
\be
\beta=\frac{2 N}{a g^2}
\eq
so that $S$ becomes the usual Yang-Mills action in the $a\to 0$ limit.
In $D=3$ dimensions, the coupling $g$ has dimensions of
$\left(\mbox{mass}\right)^{1/2}$, consistent with $\beta$ being dimensionless.
%It is sometimes convenient to use the average plaquette,
%\be
%      \plaquette_\mathrm{avg} = \frac{2}{n_L (D-1) N}
%             \sum_{x,\,\mu<\nu} \Re\,\plaquette_{\mu,\nu}(x) 
%\eq
%with normalization such that $\plaquette_\mathrm{avg}=1$
%for the bare vacuum configuration $U_\mu(x) = \mathbb{I}$.

The Wilson action is invariant under $\SU(N)$ gauge transforms:
%
\be
    U_\mu(x) \to U_\mu^\prime(x) = U_G(x) U_\mu(x) U_G^\da(x+a \hat{\mu})
         \; , \quad U_G(x) \in \SU(N) \; . \label{gauget}
\eq
%
The lattice has a total of $n_G = V_L\left(N^2-1\right)$ gauge symmetries.

\subsection{Norms}

It is useful to define a number of norms.  We define the norm
of a link field using the Frobenius norm of the logarithm,
%
\be
   \fnorm{U_\mu(x)} =
   \sqrt{\Tr M M^\dagger} \quad \mbox{where}
   \quad M = \log\left(U_\mu(x)\right)
   \label{sunorm}
\eq
and the matrix logarithm is defined in Appendix~\ref{roots}.
Note that the norm reaches its maximum value for non-identity
elements of the center $\zentrum$ of $\SU(N)$:
\be
    \fnorm{z} = 2\pi \sqrt{\frac{N-1}{N}} \; ,
     \quad z\in \zentrum,\; z\ne\mathbb{I} \; .
\eq
One can use this to define a norm for an entire
lattice configuration:
\be
   \fnorm{\config} =
   \sqrt{\frac{1}{V_L D} \sum_{\mu,x} \fnorm{U_\mu(x)}^2}
   \; . \label{latnorm2}
\eq
%
%and the distance between two lattice configurations,
%\be
%   \left\langle \config, \config^\prime \right\rangle_2 =
%   \sqrt{\frac{1}{V_L D} \sum_{\mu,x} \fnorm{U_\mu^\dagger(x)
%     U_\mu^\prime(x)}^2} \; .
%\eq
%The distance metric should not be taken too seriously, 
%since it is not gauge invarient.
%
We can also define a link norm modulo the center $\zentrum$,
\be
  \znorm{U_\mu(x)} = \min_{z\in \zentrum} \fnorm{z U_\mu(x)} \; .
\eq
That is, $\znorm{U_\mu(x)}$ is the distance
of $U_\mu(x)$ from the nearest element of $\zentrum$.
Analogous to Eqn.~(\ref{latnorm2}), we can
use $\znorm{U_\mu(x)}$ to define the associated lattice norm
\be
    \znorm{\config} = \sqrt{\frac{1}{V_L D} \sum_{\mu,x}
  \znorm{U_\mu(x)}^2}
   \; . \label{latnormz}
\eq

\section{Observables}
\label{observables}

We will use Polyakov loops and Wilson loops to monitor the
behavior of the gauge fields when moving from
the original Euclidean lattice configuration toward
a saddle-point.  The goal here is not to make precise
measurements of the string tension and glueball spectra
but to monitor the qualitative behavior for just a single
lattice configuration.

As we move toward the saddle-point of a lattice configuration,
we expect that the perturbative (quantum) fluctuations of
the gauge field to change greatly.  It is not clear how the
standard strategies for improving lattice
operators~\cite{teper_$mathrmsun$_1998,lucini_glueballs_2004}
might behave during this process.  We will not tackle that
issue in the present investigation, but will stick to the bare
(unimproved) operators.


\subsection{Polyakov loop correlators}

\begin{figure}
  % Generated in the "Bulk Polyakov loop correlator" section in "gauge.nb"
  \includegraphics{polycorr12-16-20}
  \caption{Polyakov loop correlator vs.\ loop area together with
    a fit to Eqn.~(\ref{stringmodel}) for an $\SU(3)$,
    $12\times 16 \times 20$ lattice, $\beta=28$,
    with $\left|\mathbf{r}\right|\ge 4$.
    Each color represents a different Polyakov loop direction.
    The data is from 300 lattice configurations. \label{pcorr16}}
\end{figure}

Let us define a Polyakov loop that winds once around the lattice
in the direction $\mu$ and goes through site $x$:
\be
         \phi_\mu(x)= \prod_{\rho=1}^L U_\mu(x+a \hat{\mu} \rho)
\eq
and define the correlator:
\be
\Phi_\mu(\mathbf{r}) = \frac{1}{C_\Phi(\mathbf{r})} \sum_{x, \,y,\, x^\mu = y^\mu =1}
           \delta_{\mathbf{r},\mathbf{d}(x-y)}
           \langle 0 | \Tr\left(\phi_\mu^\dagger(y)\right)
            \Tr\left(\phi_\mu(x)\right) |0\rangle
           \label{pcorr}
\eq
%
where $\mathbf{r}$ has units of $a$ and
$\mathbf{d}(x-y)$ is the minimum length positive-component
vector between $x$ and $y$, taking into account lattice periodicity
and rotation \& reflection symmetries.
The normalization constant $C_\Phi(\mathbf{r})$ is chosen such that
$\Phi_\mu(\mathbf{r})=1$ for the bare vacuum.
To obtain the string tension,
we fit the correlator $\Phi_\mu(\mathbf{r})$ to an effective
potential $F_\mu(\mathbf{r})$.
This effective potential contains the string tension $\sigma$
%plus Coulomb corrections
%(Appendix~\ref{coulomb})
plus power law corrections:
%  M. Teper, “Large N and confining flux tubes as strings -
%  a view from the lattice,”
%  Acta Phys. Polon. B 40, 3249 (2009). [arXiv:0912.3339 [hep-lat]].
%
\begin{eqnarray}
         F_\mu(\mathbf{r}) &=& c_0 \exp\left(
         -\sigma a^2 \left|\mathbf{r}\right| L_\mu
         % - c_q 2 L_\mu \log(\left|\mathbf{r}\right|)
         - c_p 2 L_\mu
         - c_1 \frac{L_\mu}{\left|\mathbf{r}\right|}
         - c_2 \frac{\left|\mathbf{r}\right|}{L_\mu} - \cdots \right)\nonumber\\
         & & + \left[\,\mbox{$2^{D-1}-1$ transverse images}\,\right]
         \; .  \label{stringmodel}
\end{eqnarray}
%
In the case of a cube-shaped lattice $L_\mu = L$, the $c_p$ term can
be absorbed into the normalization constant $c_0$ and
the $c_2$ term is indistinguishable from the string tension.
If excited states are ignored, the perimeter term $c_p 2 L_\mu$ is a
renormalization scheme-dependent self-energy,
$c_1 L_\mu/\left|\mathbf{r}\right|$ is the universal leading correction
for open strings, with $c_1 = -\pi (D-2)/24$, and
$c_2 \left|\mathbf{r}\right|/L_\mu$ is the universal correction
for closed strings, with $c_2 = -\pi (D-2)/6$; there is no
$\left|\mathbf{r}\right| L_\mu^0$
term~\cite{luscher_anomalies_1980,luscher_quark_2002,aharony_effective_2013}.
The bare Polyakov loop operator should have a small overlap with
any excited state, so one would expect this picture to hold to a
good approximation.

Since the lattices we use have a relatively small volume, we include
contributions from fields wrapping around the lattice in the plane
transverse to $\hat{\mu}$.  Thus, we sum over the $2^{D-1}$ closest
images of the Polyakov loop in the transverse plane.

When measuring $\Phi_\mu(\mathbf{r})$ for a single lattice configuration,
the link fields are re-used for each value of $\mathbf{r}$.  This means
that the loop correlators $\Phi_\mu(\mathbf{r})$ versus $\mathbf{r}$
are not statistically independent, compromising the $\chi^2$ fitting process.
To address this,
we calculate the sample covariance matrix $\mathcal{C}_\mu$ for each $\mu$:
\be
\left(\mathcal{C}_\mu\right)_{i, j} = \cov{\Phi_\mu(\mathbf{r}_i)}{\Phi_\mu(\mathbf{r}_j)} 
\eq
where $\left\{\mathbf{r}_1, \mathbf{r}_2, \ldots\right\}$ is the set of
possible transverse separations.
Then, one can fit the correlators to the effective
potential using the standard formula:
\be
\chi^2 = \sum_{\mu,i,j} \left(\Phi_\mu(\mathbf{r}_i) - F_\mu(\mathbf{r}_i)\right)
               \left(\mathcal{C}_\mu^{-1}\right)_{i,j}
               \left(\Phi_\mu(\mathbf{r}_j) - F_\mu(\mathbf{r}_j)\right) \; .
\eq

An example fit is shown in Fig.~\ref{pcorr16}.  The lattice configurations
were generated with the Chroma library~\cite{edwards_chroma_2005};
see Appendix~\ref{configurations} for details.
The string tension $\sigma a^2 = 0.0181(20)$
agrees with Teper's value for $\beta=28$,
$\sigma a^2 = 0.016210(54)$~\cite{lucini_$mathrmsun$_2002,athenodorou_sun_2017}.
The fit has $\chi^2=229$ for 195 degrees of freedom,
with the covariance matrix derived from 600 lattice configurations.
However, it should be noted that several model parameters,
$c_2$, $c_p$, and $\sigma a^2$ are highly correlated.
This example shows that inclusion of the transverse
images of the Polyakov loop are important for a good fit.

At small enough distances, the theory becomes perturbative and the
effective potential will become dominated by the Coulomb potential.
H. Meyer defines a length scale $\meyerscale$ that marks the transition from
the perturbative regime to the string-dominated
regime~\cite{meyer_static_2006},
%
% sommerScale[]*{valueError[0.402, 0.045], valueError[2.80, 0.20]}
\be
         \meyerscale \sqrt{\sigma} =  0.496(56) + \frac{3.45(25)}{N} \; .
\eq
%
Our effective potential should be valid for
$\left|\mathbf{r}\right|>\meyerscale$.  For the example shown in
Fig.~\ref{pcorr16}, $\meyerscale \approx 13 a$, substantially larger
than the cutoff $\left|\mathbf{r}\right|\ge 4$.
So, at least at this level of precision, we see that $F_\mu(\mathbf{r})$
continues to work well at substantially smaller distance scales.

\subsection{Wilson loops}

\begin{table}
  \caption{Parameters for the fit shown in Fig.~\ref{wilsonFit16}.
    \label{wilsonparameters}}
  \begin{tabular}{c|c}
    name & value \\
    \hline
  $c_0$ &  1.397(78)\\
  $\sigma a^2$ & 0.0146(20) \\
  $c_q$ &  -0.0014(43)\\
  $c_p$ &  0.0749(22)\\
  $c_1$ &  0.233(35)\\
  $c_2$ &  -0.147(26)\\
  $c_3$ &  0.091(68)\\
  $c_4$ & 0.259(39)\\
  \end{tabular}
\end{table}

\begin{figure}
  \includegraphics{wilsonFit-16}
  \caption{Wilson loop expectation value vs.\ loop size together with
    a fit to Eqn.~(\ref{wfit}) for an $\SU(3)$,
    $16^3$ lattice, $\beta=28$.  The data is from 200 lattice
    configurations.  The error bars on
    the data are too small to be seen. \label{wilsonFit16}}
\end{figure}

We define the Wilson loop operator in the $\mu,\nu$-plane using a product of
link fields lying on the perimeter of a $w_1\times w_2$ rectangle
\be
       \wilson{\mu}{\nu}{w_1}{w_2} = \frac{1}{C_W} \sum_{x}\,
         \underbrace{U_\mu(x) U_\mu(x+a \hat{\mu}) \,\cdots\, U_\nu^\dagger (x)
         }_{\mbox{$w_1\times w_2$ rectangle}}
\eq
with normalization $C_W$ chosen such that $\wilson{\mu}{\nu}{m}{n} = \mathbb{I}$
for the bare vacuum.  In this case, we can fit
$\mathcal{W} = \left\langle 0 \right|
 \Tr \wilson{\mu}{\nu}{w_1}{w_2} \left|0\right\rangle $
to an effective
potential containing a linearly confining term plus Coulomb terms
% (Appendix~\ref{coulomb})
plus power law corrections:
%
\begin{align}
  \MoveEqLeft W\left(w_1, w_2; L_\mu, L_\nu\right) =  \nonumber \\
       & c_0^\prime \exp\left[
       -\sigma a^2 w_1 w_2 
       - c_q  \left(\Theta\!\left(w_1, w_2\right) +
       \Theta\!\left(w_2, w_1\right)\right)
       - c_p  2\left(w_1+w_2\right)
     \vphantom{\frac{w_1}{w_2}}\right. \nonumber \\ & \hspace{4em} \left.
           - c_2 \left(\frac{w_1}{w_2}+\frac{w_2}{w_1}\right)
           - c_3 \left(\frac{w_1}{w_2^2}+\frac{w_2}{w_1^2}\right)
           - c_4 \left(\frac{1}{w_1^2}+\frac{1}{w_2^2}\right) - \cdots
       \right]\nonumber\\ &
      \mbox{} + c_0^\prime \exp\left[
       -\sigma a^2 (L_\mu L_\nu - w_1 w_2) -
           c_q  \left(\Theta\!\left(w_1, L_\nu - w_2\right) +
               \Theta\!\left(w_2, L_\mu - w_1\right)\right)
       \right. \nonumber \\ & \hspace{4em} \left.
       - c_p  2\left(w_1+w_2\right) 
     - c_1\left(L_1 +L_2 - w_1 -w_2\right) - \cdots
     \right]\; .  \label{wfit}
\end{align}
%
The second exponential term represents the contribution from
the {\em exterior} of the rectangle, given that the lattice
is periodic in the $\hat{\mu}$ and $\hat\nu$ directions.


An example fit is shown in Fig.~\ref{wilsonFit16}.  
The fit has $\chi^2=23$ for 13 degrees of freedom,
with the covariance matrix derived from 600 lattice configurations.
The best fit parameters are given in Table~\ref{wilsonparameters}.:
The string tension agrees with Teper's value for $\beta=28$,
$\sigma a^2 = 0.016210(54)$~\cite{lucini_$mathrmsun$_2002,athenodorou_sun_2017}.
The Coulomb term $c_q$ is not significant, but it is highly anti-correlated
with the string tension $\sigma a^2$.  The Wilson loop fit does not
do a very good job of untangling the two.


\section{Landau gauge}

In the continuum, the choice of Landau gauge $\partial_\mu A^\mu = 0$
is equivalent to minimizing $\Tr A_\mu A^\mu$ over all space.  The fact that
$\partial_\mu A^\mu = 0$ has multiple solutions (the Gribov problem) is
equivalent to the statement that $\Tr A_\mu A^\mu$, integrated over space,
has multiple local minima.~\cite{maas_more_2009}.
We are interested here in Landau gauge mainly
because it gives us, if $\Tr A_\mu A^\mu$ is small enough, the
hope of connecting the lattice
link fields to the continuum degrees of freedom.
On the lattice, this is equivalent to finding a gauge transform
that minimizes the lattice norm $\fnorm{\config}$.  In this
section, we will explore some strategies for finding this minimum.

Let us start with local minimization.
For some site $x$, we want to find a gauge transform $U_G$ on $x$
that minimizes $\fnorm{U_\mu(x)}$ for the $2D$
links connected to $x$.  Taylor expanding in the link fields,
we find, to lowest order in the fields, a gauge transform
that minimizes this norm:
%
\be
          U_G(x) = e^{- \overline{M}}
\eq
where
\be
   \overline{M} = \frac{\orelax}{2 D} \sum_\mu \log\left(U_\mu(x)\right) -
   \log(U_\mu\left(x-a \hat{\mu})\right)  \; . \label{landau}
\eq
At each site of the lattice, one can calculate $U_G(x)$ and apply it,
moving through the lattice sites in a checkerboard fashion,
first all the even sites and then all the odd sites.
This constitutes one minimization step, what we will call
a ``Landau step,'' in a
successive over-relaxation (SOR) procedure.
The relaxation factor $\orelax$ can be adjusted to speed numerical
convergence.
In practice, we find the fastest convergence for $\orelax \approx 1.7$;
however, it can become unstable for slightly higher $\orelax$.
We generally use $\orelax=1.5$ as a compromise between speed and stability.

A variation of this procedure would be to minimize
$\znorm{\config}$ instead.
That is, we minimize the norm of the link
fields modulo some element of the center of $\SU(N)$, a ``center step.''
In this case, we find a gauge transform $U_G$ on $x$
that minimizes $\znorm{U_\mu(x)}$ for the $2D$
links connected to $x$.
That is, when computing the logarithms in Eqn.~(\ref{landau}) one uses
\be
\log\left(U_\mu(x)\right) \to \log\left(z^k U_\mu(x)\right)
\, , \;\mbox{
  where $z^k\in\zentrum$ minimizes}\; \fnorm{z^k U_\mu(x)} \;.
\eq


\subsection{Maximal center gauge}

\begin{table}
  \caption{Local search for maximal center gauge,
    showing successive applications of ``center step'' to
    an $\SU(3)$ lattice configuration ($L^3=16^3$, $N=3$, $\beta=28$).
    The last three columns show the number of links closest
    to each element of the center, $z^k \in \zentrum$.
    \label{center1}}
  \begin{tabular}{c|c|c|ccc}
    iteration $\kappa$ & operation & $\znorm{\config_\kappa}$
     & $z^0=\mathbb{I}$ & $z^1$ & $z^2$ \\
    \hline
    0 & original config & 2.377 & 4101 & 4121 & 4066\\
    1 & center step     & 1.811 & 4090 & 4145 & 4053\\
    2 & center step     & 1.625 & 4085 & 4147 & 4056\\
    3 & center step     & 1.568 & 4085 & 4150 & 4053\\
    \vdots & \vdots & \vdots & \multicolumn{3}{c}{\vdots}\\
    11 & center step    & 1.503 & 4090 & 4144 & 4054\\
    12 & center step    & 1.501 & 4090 & 4144 & 4054\\
  \end{tabular}
\end{table}

\begin{table}
  \caption{Successive applications of $\zentrum$-axial gauge
    and ``center step'' to the lattice configuration
    used in Table~\ref{center1}.
    The last three columns show the number of links closest
    to $z^k \in \zentrum$.
    \label{center2}}
  \begin{tabular}{c|c|c|ccc}
    iteration $\kappa$ & operation & $\znorm{\config_\kappa}$
     & $z^0=\mathbb{I}$ & $z^1$ & $z^2$ \\
    \hline
    0 & original config              & 2.377  & 4101 & 4121 & 4066\\
    1 & $\zentrum$-axial gauge $\mu=1$& 1.950  & 4140 & 4053 & 4095\\
    2 & $\zentrum$-axial gauge $\mu=2$& 1.664  & 4189 & 4024 & 4075\\
    3 & center step                  & 1.268  & 4191 & 4023 & 4074\\
    4 & $\zentrum$-axial gauge $\mu=3$& 1.553  & 4116 & 4073 & 4099\\
    5 & center step                  & 1.120  & 4121 & 4076 & 4091\\
    6 & center step                  & 0.9565 & 4121 & 4080 & 4087\\
    7 & center step                  & 0.9022 & 4115 & 4082 & 4091\\
    \vdots & \vdots & \vdots & \multicolumn{3}{c}{\vdots}\\
    19 & center step                 & 0.8636 & 4117 & 4082 & 4089\\
    20 & center step                 & 0.8632 & 4117 & 4082 & 4089\\
  \end{tabular}
  \end{table}

\begin{table}
  \caption{Successive applications of axial gauge
    and ``center step'' to the lattice configuration
    used in Table~\ref{center1}.
    The last three columns show the number of links closest
    to $z^k \in \zentrum$.
    \label{center3}}
  \begin{tabular}{c|c|c|ccc}
    iteration $\kappa$ & operation & $\znorm{\config_\kappa}$
     & $z^0=\mathbb{I}$ & $z^1$ & $z^2$ \\
    \hline
    0 & original config    & 2.377  & 4101 & 4121 & 4066\\
    1 & axial gauge $\mu=1$& 1.945  & 6826 & 2805 & 2657\\
    2 & axial gauge $\mu=2$& 1.626  & 9668 & 1450 & 1170\\
    3 & center step        & 1.244  & 9674 & 1450 & 1164\\
    4 & axial gauge $\mu=3$& 1.455  & 11369 & 386 & 533\\
    5 & center step        & 1.035  & 11409 & 363 & 516\\
    6 & center step        & 0.8543 & 11455 & 340 & 493\\
    7 & center step        & 0.7887 & 11471 & 334 & 483\\
    \vdots & \vdots & \vdots & \multicolumn{3}{c}{\vdots}\\
    19 & center step       & 0.7219 & 11498 & 317 & 473\\
    20 & center step       & 0.7214 & 11498 & 317 & 473\\
  \end{tabular}
  \end{table}

\begin{table}
  \caption{Successive applications of axial gauge,
    ``Landau step,'' and ``center step'' to the lattice configuration
    used in Table~\ref{center1}.
    The last three columns show the number of links closest
    to $z^k \in \zentrum$.
    \label{center4}}
  \begin{tabular}{c|c|c|ccc}
    iteration $\kappa$ & operation & $\znorm{\config_\kappa}$
     & $z^0=\mathbb{I}$ & $z^1$ & $z^2$ \\
    \hline
    0 & original config    & 2.377  & 4101 & 4121 & 4066\\
    1 & axial gauge $\mu=1$& 1.945  & 6826 & 2805 & 2657\\
    2 & axial gauge $\mu=2$& 1.626  & 9668 & 1450 & 1170\\
    3 & center step        & 1.244  & 9674 & 1450 & 1164\\
    4 & axial gauge $\mu=3$& 1.455  & 11369 & 386 & 533\\
    5 & Landau step        & 1.018  & 12271 &   8 &   9\\
    6 & center step        & 0.7231 & 12268 &   9 &  11\\
    7 & Landau step        & 0.6009 & 12287 &   0 &   1\\
    8 & center step        & 0.5426 & 12286 &   1 &   1\\
    9 & center step        & 0.5128 & 12286 &   1 &   1\\
    10 & center step       & 0.4952 & 12286 &   1 &   1\\
    11 & center step       & 0.4835 & 12286 &   1 &   1\\
    12 & center step       & 0.4751 & 12287 &   1 &   0\\
  \end{tabular}
  \end{table}

Although maximal center gauge~\cite{del_debbio_center_1997}
has been an important topic in the
literature~\cite{greensite_confinement_2003},
we will not consider it separately from Landau gauge.
This section explains why and introduces our best algorithm
for finding Landau gauge.

On the lattice, maximal center gauge corresponds
to a gauge transform that minimizes $\znorm{\config}$. 
If we apply some number of ``center steps,'' defined above,
to a lattice configuration, we find quick convergence to some local
minimum of $\znorm{\config}$ with
link fields distributed fairly evenly across $\zentrum$.
See Table~\ref{center1} for an example.

However, this procedure fails to remove long-range field fluctuations
associated with the random choice of gauge.  We can remove
such fluctuations by another gauge choice:  axial gauge
$\partial_\mu A_\mu = 0$ (no sum on $\mu$).
That is, for direction $\mu$, we choose a gauge such that $U_\mu(x)$
is replaced by the $L_\mu^\mathrm{th}$ root of the associated Polyakov loop,
$U_\mu(x) \to \phi_\mu(x)^{1/L_\mu}$ where the root is defined in
Appendix~\ref{roots}.
% While not a complete gauge fixing, this gauge choice does
% remove most long-scale fluctuations caused by the
% gauge choice itself.
This gauge choice minimizes $\fnorm{U_\mu(x)}$
on the link fields in $\phi_\mu(x)$.  One might
object that this minimization is not appropriate for maximal center gauge
since it preferentially minimizes the link fields
with respect to $\mathbb{I}$ and not some other element of $\zentrum$.
This leads us to a variation which we will call ``$\zentrum$-axial gauge''
where we instead minimize $\znorm{U_\mu(x)}$
for the link fields in $\phi_\mu(x)$.  In Table~\ref{center2},
we apply this gauge transform in each direction before applying
some number of local minimizations.  One can see that this
results in a substantially improved lattice norm relative to
Table~\ref{center1}.  We also see that the link fields
remain relatively evenly distributed over the elements of $\zentrum$.

But is $\zentrum$-axial gauge actually an improvement over
conventional axial gauge?
In Table~\ref{center3}, we apply axial gauge in each
direction before applying some number of ``center steps.''
We see that this produces a lattice norm that is, in fact,
substantially lower than before.  We also can see, since
axial gauge preferentially minimizes $U_\mu(x)$ with
respect to $\mathbb{I}$, substantially more links lie near
the identity element.

Can we do better? Let us include some number of ``Landau steps''
in the procedure; see Table~\ref{center4}.  (The strategy shown
here was the end result of a an extensive Monte Carlo study that
included a number of other gauge fixing strategies, as well.)
We see a substantially reduced lattice norm, relative to
Table~\ref{center3}.  We also see that virtually all the
links are closest to identity element.  Thus, this procedure
minimizes both $\znorm{\config}$ and $\fnorm{\config}$.
We won't give details here, but similar calculations for
$\SU(2)$ and $\SU(4)$ produce comparable results.
In conclusion, we find evidence that, for $D=3$,
``absolute'' maximal center gauge is identical to ``absolute''
Landau gauge (using the language of Ref.~\cite{maas_more_2009});
Table~\ref{center4} gives
an effective strategy for finding the associated minimum.

This analysis casts doubt on the various lattice studies involving
maximal center gauge and center projections~\cite{
del_debbio_center_1997,del_debbio_detection_1998,de_forcrand_relevance_1999}.
Presumably, in these studies, relatively large local minima of
$\znorm{\config}$ were being investigated
rather than configurations near a global minimum.


\subsection{Newton's method minimization}

We introduce a final strategy for finding Landau gauge:
Newton's method.  Here, we find a gauge transform, Eqn.~(\ref{gauget}),
that minimizes the lattice norm squared $\sum_{\mu,x}\fnorm{U_\mu^\prime(x)}^2$
as a function of $B_a(x)$,
where $U_G(x) = \exp\left(i B_a(x)\, T_a\right)$.
We construct the the gradient and Hessian matrices
and solve the associated linear system to find a local minimum of
$\fnorm{\config}^2$.
Details of this method can be found in Appendix~\ref{newtonNorm}.

How, does this method compare with the much simpler ``Landau step'' defined
earlier?  In a ``Landau step'' we minimize the fields in the tangent space
of the gauge group, which can be justified if the gauge fields are small.
In Newton's method, we assume the gauge transform is small, but
assume nothing about the gauge fields.

In practice, we find that Newton's method performs well in the
vicinity of a local minimum of $\fnorm{\config}^2$.
The convergence of ``Landau step'' is slower, but is less susceptible
to getting stuck in a non-optimal local minimum.  In addition,
the ``Landau step'' requires significantly fewer computational resources.

Thus, when minimizing $\fnorm{\config}^2$, we will mostly rely
on ``Landau steps'' and ``center steps,`` as shown in Table~\ref{center4},
and apply 2-3 Newton's method steps at the very end,
when already close to the minimum.

\section{Finding the saddle point}
\label{saddlePoint}

In the following, we will introduce an iterative
procedure---Newton's method---for finding the saddle-point
of the action associated with a single lattice configuration.

\subsection{Shifts of the gauge fields}

\begin{figure}
  \[
  \begin{array}{l}
    \mbox{} \hspace{0.2in} x \hspace{0.3in} \sqrt{U_\mu(x)}
    \hspace{0.25in} e^{i w_{a}(\mu, x) T_a}
    \hspace{0.25in} \sqrt{U_\mu(x)} \hspace{0.3in} x+a\hat{\mu} \\[-0.3in]
    %
    %  Graphic generated in Mathematica file ``gauge.nb''
    %
  \includegraphics{link}
  \end{array}
  \]
  \caption{One can think of a shift as applying a color rotation
    to the middle of a lattice link. \label{shift}}
\end{figure}

In order to find a saddle point, it is convenient to modify
the link fields $U_\mu(x)$ in a manner that maintains lattice symmetries.
We define a ``shift'' for each link as
\be
  U_\mu(x) \to \sqrt{U_\mu(x)}\, e^{i w_{a}(\mu, x) T_a}\, \sqrt{U_\mu(x)} \; ,
    \label{shifts}
\eq
where $T_a$ are generators of $\SU(N)$, with normalization
$\delta_{a,b} = 2 \Tr(T_a T_b)$ and the square root is defined
in Appendix~\ref{roots}.
We can then take partial derivatives of $N S/\beta$ with
respect to the $n_L$ shifts $\mathbf{w} = \left(w_1(1, x_1), \ldots\right)$
to obtain the gradient vector $\mathbf{q}$ and Hessian matrix
$H$.  The gradient and Hessian can be used to estimate the
position of the nearest saddle point.

Let us consider the gradient for the link $U_\mu(x)$.
The gradient will involve the $2 (D-1)$ plaquettes that include that link.
Let $F_\mu(x)$ be the associated sum of staples; then
\be
   \frac{N}{\beta} \frac{\partial S}{\partial w_a(\mu, x)} =
   \Im\Tr\left(F_\mu(x) \sqrt{U_\mu(x)} T_a \sqrt{U_\mu(x)}\right) \; .
   \label{grad}
\eq
We can use Eqn.~(\ref{grad}) to construct the gradient vector $\mathbf{q}$.
Likewise, one can calculate the $n_L\times n_L$ Hessian matrix
$H$ using
\be
      \frac{N}{\beta} \frac{\partial^2 S}{\partial w_a(\mu, x)\, \partial w_b(\nu, y)}
\eq
which is nonzero when there is a plaquette which contains both
$U_\mu(x)$ and $U_\nu(y)$.  For example, for $x=y$,
\be
      \frac{N}{\beta} \frac{\partial^2 S}{\partial w_a(\mu, x)\, \partial w_b(\nu, x)} = \frac{1}{2}
   \Re\Tr\left(F_\mu(x) \sqrt{U_\mu(x)} \left(T_a T_b + T_b T_s\right) \sqrt{U_\mu(x)}\right) \; .
   \label{diaghess}
\eq
The $x \neq y$ terms of the Hessian are constructed in a similar fashion.

Some shifts $\mathbf{w}$ correspond to gauge transforms.
As we shall see, such shifts are problematic for any saddle
point search, since the action is constant in those directions.
To address this issue, we determine the set of shifts that are
equivalent to infinitesimal gauge transforms.  Consider a gauge transform
at site $x$; its action on link $U_\mu(x)$ is:
\be
U_\mu(x) \to e^{i C_a(x) T_a} U_\mu(x) = U_\mu(x) + i C_a(x) T_a U_\mu(x) +
       \mathrm{O}\!\left(C_a^2\right)
\eq
But this is equal to some infinitesimal shift $w_a(\mu, x)$ on
the same link $U_\mu(x)$:
\be
U_\mu(x) \to U_\mu(x) + i w_a(\mu,x) \sqrt{U_\mu(x)}T_a \sqrt{U_\mu(x)} +
       \mathrm{O}\!\left(w_a^2\right)
\eq
%
Equating the two and taking the trace:
\be
w_a(\mu,x) = 2 C_b(x) \Tr\left(\sqrt{U_\mu(x)} T_a
                     \sqrt{U_\mu^\da(x)} T_b\right) \; . \label{gs}
\eq
Using Eqn.~(\ref{gs}), one can construct an $n_G \times n_L$ matrix
$G$ that relates shifts to infinitesimal gauge transforms
at each lattice site.
Since the action is invariant under gauge transforms, $G \mathbf{q} = 0$.
However, since the Hessian represents the quadratic term, $G H G^T \neq 0$.

Since we are using the Hessian, we are approximating
the action $S$ as being quadratic in $\mathbf{w}$.  If $\mathbf{w}$
is too large, the quadratic approximation is no longer valid.
Thus, we need to monitor the size of the shifts.
Using Eqn.~(\ref{sunorm}), we will define two norms for the shifts:
a lattice-wide $\infty$-norm,
\be
\left\lVert \mathbf{w}\right\rVert_{\mathrm{max}} =
     \max_{x,\mu} \fnorm{e^{i w_{a}(\mu, x) T_a}}
     = \max_{x,\mu} \sqrt{\frac{1}{2}\sum_a w_a^2(\mu, x)}
\eq
as well as a lattice-wide Euclidean norm, averaged over links
\be
\left\lVert \mathbf{w}\right\rVert_2 =
      \sqrt{\frac{1}{V_L D} \sum_{x, \mu}
        \fnorm{e^{i w_{a}(\mu, x) T_a}}^2}
     = \sqrt{\frac{1}{2 V_L D} \sum_{x, \mu, a} w_a^2(\mu, x)}
        \; .  \label{shiftsize}
\eq

\subsection{Calculating the saddle point}
\label{saddle}

\begin{figure}
\includegraphics{eigenAll}
\caption{Histogram of the eigenvalues of $H$ for a $6^3$ lattice,
  $\beta = 8.175$, $N=3$.  We see a peak around zero due to the
  $\SU(3)$ gauge symmetry.
  \label{eigenAll}}
\end{figure}

\begin{figure}
\includegraphics{shiftPairs}
\caption{Scatter plot of eigenvalue-gradient pairs of $H^\prime$
  for a $6^3$ lattice, $\beta = 8.175$, $N=3$.  The pairs above the
  v-shaped wedge correspond to large shifts and will dominate
  the solution to the linear system, Eqn.~(\ref{linear2}).
  \label{shiftPairs}}
\end{figure}

In principle, to find the nearest saddle point, one would
find $H$ and $\mathbf{q}$, solve the linear system
\be
    H \mathbf{y} = \mathbf{q} \label{linear1}
\eq
and shift the gauge fields using $\mathbf{w} = -\mathbf{y}$.
Since the action $S$ is not quadratic in $\mathbf{w}$, one
would then iterate this process to find the actual saddle point
(Newton's method).

However, in practice, this procedure does not work.  To see why, let us
look at the eigenvalues of $H$.  As seen in Figure~\ref{eigenAll},
we see that the eigenvalues are strongly peaked around zero,
due to the gauge symmetry of the theory.  Thus, the linear
system (\ref{linear1}) will be nearly singular and
the solution will be numerically unstable.  We must be more careful.

To address this issue, we find the null-space $P_G$ of the gauge
transform matrix $G$,
\be
G P_G^T = 0 \; , \quad P_G P_G^T = \mathbb{I}\; .
\eq
Then we project $H$ and $\mathbf{q}$ onto
the subspace of non-gauge-transform shifts:
\be
         H^\prime = P_G H P_G^T \quad \mbox{and} \quad
         \mathbf{q}^\prime = P_G \mathbf{q} \; .
\eq
Then we can solve the linear system
\be
   H^\prime \mathbf{y}^\prime = \mathbf{q}^\prime \label{linear2}
\eq
and set $\mathbf{w} = - P_G^T \mathbf{y}^\prime$.  However, this is
still insufficient to yield a well-behaved linear system.

To better understand this issue, let us define the eigenvalues
$\left(\heigen_1, \heigen_2, \ldots\right)$ and
eigenvectors $V=\left(\mathbf{v}_1, \mathbf{v}_2, \ldots\right)$
of $H^\prime$ and express Eqn.~(\ref{linear2}) in the eigenspace of $H^\prime$:
\be
\begin{pmatrix}
    \heigen_1 & & \\
    & \heigen_2 & \\
    & & \ddots  \end{pmatrix} \overline{\mathbf{y}} =
  \overline{\mathbf{q}} \label{linear3}
\eq
where $\overline{\mathbf{y}} = V \mathbf{y}^\prime$ and
$\overline{\mathbf{q}}  = V \mathbf{q}^\prime$.
The solution is simply
\be
    \overline{y}_i = \frac{\overline{q}_i}{\heigen_i} \; ;
\eq
see Fig.~\ref{shiftPairs} for an example. 
For any case where $\left|\heigen_i\right|\ll \left|\overline{q}_i\right|$,
there will be a large shift that will dominate the solution
to the linear system.
Thus, we impose cutoffs $\Lambda_2$ and $\Lambda_\mathrm{max}$:
\begin{eqnarray}
    \sqrt{2} \left|\overline{q}_i\right|\left\lVert P_G^T \mathbf{v}_i\right\rVert_2
     &<& \left|\heigen_i\right|\,\Lambda_2 \label{lambda2} \\
    \sqrt{2} \left|\overline{q}_i\right|
      \left\lVert P_G^T \mathbf{v}_i\right\rVert_\mathrm{max}
    &<& \left|\heigen_i\right|\,\Lambda_\mathrm{max} \; .
\end{eqnarray}
%
and set $\overline{y}_i=0$ whenever a cutoff is violated.
That is, we remove any eigenpairs inside a v-shaped
region like the one shown in Fig.~\ref{shiftPairs}.  

Finally, since the gauge symmetry of the action is non-linear,
we do not want to shift the fields by too much before recalculating
the matrix $G$.  
Thus, it is useful to add a damping factor $\gamma$ and
further rescale by $\Lambda_s$ when the shift would
otherwise be large:
\be
  \mathbf{w} = - \gamma \mathbf{y} \min\left(1, \frac{\Lambda_s}{
    \sqrt{2} \left\lVert \mathbf{y}\right\rVert_\mathrm{max}}\right) \; .
\eq
Typically, $\Lambda_s = 1$ or $2$ is used.

Armed with these three tools:  removing infinitesimal-gauge-transform shifts,
imposing the $\Lambda$ cutoffs, and using the damping factor $\gamma$,
we can apply the shift iteratively and find saddle-point configurations
that are of physical interest.

For a $20^3$ lattice, $\SU(3)$, the Hessian is a
$192,000 \times 192,000$ matrix with almost $20\times 10^6$
nonzero matrix elements. Thus, the numerical calculations
must be carried out using sparse matrix techniques;
details are discussed in Appendix~\ref{krylov}.


\subsection{Diagonal blocks of the Hessian and cooling}

The full Newton's method calculation discussed above is numerically
challenging.
However, if we ignore the off-diagonal color blocks of the
Hessian (that is, we only use Eqn.~(\ref{diaghess})),
then the numerical calculation becomes much simpler.  Since this truncation
is {\em ad hoc}, it must be determined empirically whether it
has any utility.  In this case, links in each direction
are updated in a checkerboard fashion: first the
$\mu=1$, $x$ even links, then $\mu=1$, $x$ odd links,
$\mu=2$, $x$ even links, {\em et cetera}.  We will refer to this
as the ``single-link saddle-point method.''


This single-link saddle point method is reminiscent of the
approach used in a number of investigations of ``lattice
cooling''~\cite{trottier_exploring_1994,teper_cooling_1994,
  gonzalez-arroyo_gauge_1995,gonzalez-arroyo_classical_1996,
  bonati_comparison_2014}.
In lattice cooling, one attempts to find a local minimum
of the action by minimizing the action for
each link separately, iterating over all of the links of the lattice.
In the folliwng, we will compare the lattice cooling approach to
a search for the nearest saddle-point.  Our algorithm for performing
the link-wise minimization is discussed in Appendix~\ref{linkmin}.
As with the single-link saddle-point method, links in each direction
are updated in a checkerboard fashion:  first the even links and
then the odd links.

\subsection{Trajectories}

\begin{figure}
\includegraphics{"avgplaquette-3-4-12-16-20"}
\hfill
\includegraphics{"norm-3-4-12-16-20"}
\caption{Trajectories for $\SU(4)$, $12 \times 16 \times 20$ lattice,
  $\beta=48$. Trajectory ``5-a'' was computed with $\gamma = 0.1$ and
  cutoffs $\Lambda_2=0.02$, $\lambda_\mathrm{max}=0.04$, $\Lambda_s = 1$;
  28 steps.
  ``15-a'' starts with a different lattice configuration but uses the
   same parameters; 35 steps.
   ``5-sb'' uses the same starting configuration as ``5-a`` but steps
   are calculated with the single-link saddle-point method,
     $\gamma=0.1$; 75 steps, every 5th step is plotted.
   ``5-sc'' uses the same starting configuration as ``5-a`` but steps
   are calculated with lattice cooling,
     $\gamma=0.1$; 50 steps, every 5th step is plotted.
    \label{trajectory3}}
\end{figure}


\begin{figure}[p]
\includegraphics{"stringmodel-3-4-12-16-20-a2sigma"}\hfill
\includegraphics{"stringmodel-3-4-12-16-20-c0"}\\
\vspace{3ex}
\includegraphics{"stringmodel-3-4-12-16-20-cCoulomb"}\hfill
\includegraphics{"stringmodel-3-4-12-16-20-cPerimeter"}\\
\vspace{3ex}
\includegraphics{"stringmodel-3-4-12-16-20-chiSquared"}
\caption{$\chi^2$ fit of polyakov loop correlators to $F_\mu(\mathbf{r})$,
  Eqn.~(\ref{stringmodel}),
  for points along the four trajectories
  shown in in Fig.~\ref{trajectory3}.
  The covariance matrix used in the fit was
  constructed from 1100 lattice configurations.
  Note that error bars generally decrease as short-length-scale
  fluctuations are removed from the fields.  \label{traj4b}
 }
\end{figure}

% Allow any full-page figure to be placed just after
% the end of the current page.
\afterpage{\clearpage}

By applying the procedure discussed in Section~\ref{saddle} iteratively,
we produce a trajectory of lattice configurations $\config_0$,
$\config_1$, \ldots in configuration space.  For lattice configuration
$\config_\kappa$, we define
$\Delta_\kappa$ to be the total distance traveled through configuration
space
\be
\Delta_\kappa = \sum_{\rho=1}^\kappa \left\lVert \mathbf{w}_\rho\right\rVert_2
\eq
where $\mathbf{w}_\rho$ is the shift used to generate $\config_\rho$
from $\config_{\rho-1}$ and the norm is from Eqn.~(\ref{shiftsize}).


In Figs.~\ref{trajectory3} and \ref{traj4b}, we show three
trajectories for $\SU(4)$ in terms of $\Delta_\kappa$.  Ideally,
if the Master Field picture is correct, we should see, as
one moves along the trajectory, that observables that persist
in the large $N$ limit, like the string tension and the
glueball spectrum, should persist, at least approximately,
after the short length scale (perturbative) fluctuations are removed.

For finite $N$ and $a$, we don't expect this picture to
hold exactly, nor do we expect the saddle points to be
exact. Similar to the studies of monopoles on the lattice
% ~\cite{},
we expect there to be some point where the calculation breaks
down:  for instance $c_0$ goes to zero (correlations disappear)
or the string tension blows up.


and the string tension  $\sigma a^2$.
Lucini and Teper report a string tension $\sigma a^2=0.016282(76)$
for $N=3$, and $\beta=28$ (Ref.~\cite{lucini_$mathrmsun$_2002}, Table~5).
For the initial configuration $\config_0$ used in Fig.\ref{trajectory3},
we measure $\sigma a^2=0.0155(56)$;
with errors that remain about that size for $\config_\kappa$, $\kappa>0$.
For $\config_0$, $\plaquette_\mathrm{avg}=0.902$ and the
action is dominated by perturbative fluctuations of the gauge fields.
One can use the string tension to
estimate $\plaquette_\mathrm{avg}$ where we expect the action to become
dominated by the string-like behavior of the theory:
\be
\plaquette_\mathrm{avg} \gtrsim  1 - \sigma a^2 = 0.984 \; .
\eq


Ideally, $\plaquette_\mathrm{avg}$ will increase steadily and the string tension will remain relatively constant as we apply our iterative procedure.
This is illustrated by the trajectory {\bf 2d}.
In cases where $\gamma$ or $\Lambda_2$ is too large, we find that the
string tension does not remain stable, as illustrated by
trajectories {\bf 2b} and {\bf 2c}.

We also have investigated an ad-hoc procedure where
the off-diagonal color blocks of $H$ are dropped (that is, the
saddle point search is applied to each link individually).
This procedure is much easier to apply numerically.
As shown in trajectory {\bf lc},
we see that that the string tension can remain relatively stable,
but $\config_\kappa$ eventually degrades to the trivial vacuum.


\begin{figure}
\includegraphics{"wilsonvalues-3-4-12-16-20"}
\caption{Wilson loop expectation values along the ``5-a'' trajectory
 shown in Fig.~\ref{trajectory3}.}
\end{figure}

{\bf  To do:
  finish $\SU(3)$, $L^3=20^3$ calculation and
  add results for $\SU(4)$}

\subsection{gauge invariance}

{\bf Either give an example or delete }

The procedure described above is not gauge invariant.
However, in our numerical calculations we do see approximate
gauge invariance in the following
sense.  Starting with a gauge configuration $\config_0$ generated
by Euclidean lattice Monte Carlo, one can
use our procedure to generate a trajectory of configurations
$\config_1$, $\config_2$, \ldots
with distances $\Delta_1$, $\Delta_2$, \ldots.
If one first performs a gauge transform on the initial
configuration $\config_0 \to \config_0^\prime$ and then generates a
new trajectory $\config_1^\prime$, $\config_2^\prime$, \ldots
with distance $\Delta_\kappa^\prime$,
gauge invariant observables
measured for $\config_\kappa$ and $\config_\kappa^\prime$ are approximately equal.
Moreover,  $\Delta_\kappa \approx \Delta_\kappa^\prime$.


\subsection{Spectrum}

\begin{figure}
   configuration $\config_0$\\
   \includegraphics{corrOriginal}\\
   configuration $\config_{10}$\\
   \includegraphics{corr2d10}\\
   configuration $\config_{20}$\\
   \includegraphics{corr2d20Even}
\caption{Correlations of the even plaquette operator for several points
  along trajectory {\bf 2d} shown in Fig.~\ref{trajectory3}.
  Also shown is an exponential corresponding to Teper's value
  for the lowest $0^{++}$ glueball.
  \label{corr2dEven}}
\end{figure}

\begin{figure}
   \includegraphics{corr2d20Odd}
   \caption{Correlations of the odd plaquette operator for
     lattice configuration
     $\config_{20}$ on trajectory {\bf 2d} shown in Fig.~\ref{trajectory3}.
  Also shown is an exponential corresponding to Teper's value
  for the lowest $0^{--}$ glueball.
  \label{corr2dOdd}}
\end{figure}

\begin{figure}
   \includegraphics{corrLink995}
   \caption{Correlations of the even plaquette operator for
     the $\plaquette_\mathrm{avg} \approx 0.995$ lattice configuration
     on trajectory {\bf lc} shown in Fig.~\ref{trajectory3}.
    Also shown is an exponential corresponding to Teper's value
     for the lowest $0^{++}$ glueball.
     We do not see the desired exponential behavior.
  \label{corrLink995}}
\end{figure}

We can look at other properties of the gauge configuration
as we move along the trajectory.  In particular, let us look
at the correlators of the charge conjugation even plaquette operator
\be
  \left\langle
  \left(\Re \plaquette_{\mu,\nu}(x) - \plaquette_\mathrm{avg}\right)
  \left(\Re \plaquette_{\alpha,\beta}(y) - \plaquette_\mathrm{avg}\right)
  \right\rangle
\eq
as a function of the separation of the two plaquettes
\be
\mathrm{separation} =
        \left\lVert \left(x+a \hat{\mu}/2+ a\hat{\nu}/2\right) -
             \left(y+a \hat{\alpha}/2+ a\hat{\beta}/2\right)
             \right\rVert \; .
\eq
This correlator couples to the $0^{++}$ glueball states.

In Fig.~\ref{corr2dEven}, we plot the correlation for various relative
orientations of the two plaquettes for several points along the
best trajectory in~\ref{trajectory3}.  We also plot an exponential
corresponding to Teper's value for the $0^{++}$ glueball,
mass $0.5517(38)/a$ for $\beta=28$, $L^3=23^3$~\cite{teper_$mathrmsun$_1998}.
The exponential only demonstrates that there is qualitative agreement.
Since the correlator also has contributions from the excited $0^{++}$
states, a quantitative fit would require a more extensive analysis,
with additional operators in the correlation function.  However, one
would expect the single plaquette operator to couple very strongly
to the lowest state and not so strongly to the excited states.

Likewise, we can look at the charge conjugation odd correlator
\be
  \left\langle
  \left(\Im \plaquette_{\mu,\nu}(x) - \plaquette_\mathrm{avg}\right)
  \left(\Im \plaquette_{\alpha,\beta}(y) - \plaquette_\mathrm{avg}\right)
  \right\rangle   \; ,
\eq
corresponding to the $0^{--}$ glueball states; see Fig.~\ref{corr2dOdd}.
Also shown is an exponential corresponding to the lowest $0^{--}$ state,
mass $0.8133(57)/a$ \cite{teper_$mathrmsun$_1998}.

In the case where the off-diagonal color blocks of $H$ are dropped,
we do not see the desired exponential behavior in the correlator,
Fig.~\ref{corrLink995}.

\section{Gauge field correlators}

In this section, we examine gauge correlations of
the gauge fields.
Since the lattice norm $\fnorm{\config}$ is invariant under global
color rotations, the link fields in Landau gauge are physical up
to global color rotations.  One can construct physically meaningful
field correlators from color singlet combinations of the gauge fields
in minimum $\fnorm{\config}$ gauge.
We will express the correlators in terms of the tangent fields
\be
i \mathbf{A}_\mu(x) = i A_{\mu, a}\left(x\right) T_a =
     \log\left(U_\mu(x)\right)
\eq
where the logarithm is defined in Appendix~\ref{roots}.

\begin{table} \caption{2-point color singlet correlators. \label{field2}}
  \begin{tabular}{ccc}
    \framebox{2t} & $\displaystyle \sum_{\mu,\, x,\, \mathbf{R} \perp \hat{\mu}}
    \Tr\mathbf{A}_\mu(x) \mathbf{A}_\mu(x+\mathbf{R})$ &
    \raisebox{-2ex}{\includegraphics{field2t}} \\    
    \framebox{$2 \ell$} & $\displaystyle \sum_{\mu,\, x,\, R}
    \Tr\mathbf{A}_\mu(x) \mathbf{A}_\mu(x+R \hat{\mu})$ &
    \raisebox{-2ex}{\includegraphics{field2l}}
  \end{tabular}
\end{table}

\begin{table} \caption{3-point color singlet correlators. \label{field3}}
  \begin{tabular}{ccc}
    \framebox{3p$\ell$} & $\displaystyle \sum_{\mu,\, x,\, 0\le r \le R}
     A_{\mu,a}(x) A_{\mu,b}(x+r\hat{\mu})A_{\mu,c}(x+R\hat{\mu}) f_{a,b,c}$ &
    \raisebox{-2ex}{\includegraphics{field3pl}} \\    
    \framebox{3pt1} & $\begin{array}{r}
      \displaystyle \sum_{\substack{\mu,\, \nu,\, x,\\ 0\le r \le R,\,
      \hat{\mu} \perp\hat{\nu}}} \left[
      A_{\mu,a}(x) A_{\nu,b}(x+r\hat{\mu})A_{\nu,c}(x+R\hat{\mu}) + \mbox{}\right. \\
      \displaystyle\left. A_{\nu,a}(x) A_{\nu,b}(x+r\hat{\mu})A_{\mu,c}(x+R\hat{\mu}) \right] f_{a,b,c} \end{array}$ &
    \raisebox{-2ex}{\includegraphics{field3pt1}} \\
    \framebox{3pt2} & $
      \displaystyle \sum_{\substack{\mu,\, \nu,\, x,\\ 0\le r \le R,\,
      \hat{\mu} \perp\hat{\nu}}}
      A_{\nu,a}(x) A_{\mu,b}(x+r\hat{\mu})A_{\nu,c}(x+R\hat{\mu}) f_{a,b,c}$ &
    \raisebox{-2ex}{\includegraphics{field3pt2}} \\    
    \framebox{3m1} & $
      \displaystyle \sum_{\substack{\mu, \nu,\, x,\\ 0\le r,\, 0 \le R,\,
      \hat{\mu} \perp\hat{\nu}}}
      A_{\mu,a}(x) A_{\mu,b}\left(x+\frac{R}{2} \hat{\mu}+r\hat{\nu}\right)
       A_{\mu,c}(x+R\hat{\mu}) f_{a,b,c}$ &
    \raisebox{-2ex}{\includegraphics{field3m1}} \\    
    \framebox{3m2} & $
      \displaystyle \sum_{\substack{\mu, \nu,\, x,\\ 0\le r,\, 0 \le R,\,
      \hat{\mu} \perp\hat{\nu}}}
      A_{\mu,a}(x) A_{\mu,b}\left(x+\frac{R}{2} \hat{\mu}+r\hat{\nu}\right)
       A_{\nu,c}(x+R\hat{\mu}) f_{a,b,c}$ &
    \raisebox{-2ex}{\includegraphics{field3m2}} \\    
  \end{tabular}
\end{table}

\section{Conclusions}

The master field idea is a statement about the $N\to \infty$ limit
of $\SU(N)$ gauge theory.  However, if finite $N$ is similar
to the $N\to\infty$ limit, we should expect that properties of
the master field emerge, approximately, at finite $N$.
As an example of this, we
see that a saddle point of a single lattice configuration
contains the necessary physics, at least approximately, to produce the
correct low-energy spectrum of the theory, string tensions,
and Wilson loop expectation values.

In addition, we see 

\vspace{10mm}
\noindent {Acknowledgments}:

\appendix

\section{Roots of $\SU(N)$ matrices}
\label{roots}

In general, for matrix $U_c$ belonging to a connected Lie group,
the matrix power $U_c^t$, $t\in[0,1]$ should be a
smooth map from $[0,1]$ onto the group manifold where
$U_c^0=\mathbb{I}$, $U_c^1=U_c$, and $U_c^s U_c^t = U_c^{s+t}$;
the image of the map should have minimal length.
A unitary matrix $U$ can be factored as
%
\be
U = V \begin{pmatrix}
    \mathrm{e}^{i \lambda_1} & & &\\
    & \mathrm{e}^{i \lambda_2} & &\\
    & & \ddots & \\
    & & & \mathrm{e}^{i \lambda_N}\end{pmatrix} V^\da
\eq
%
where $V$ is a unitary matrix and $-\pi < \lambda_i < \pi$.
In the special case $U \in \SU(N)$, the $\det(U)=1$ condition implies that
\be
\sum_j \lambda_j = 2 \pi m\;, \quad m\in\integer \;.
\eq
When taking matrix roots, cases where $m\neq 0$ present a
difficulty.  However, one can always shift $\lambda_j$ by
multiples of $2\pi$:
\be
\lambda_j \to \lambda_j^\prime = \lambda_j + 2 \pi n_j\;,\quad
n_j\in\integer \; ,
\eq
such that $\sum_j \lambda_j^\prime = 0$ and
$\lambda_j^\prime - \lambda_k^\prime \le 2 \pi$.
%
\begin{figure}
\includegraphics{hex3}
\caption{In the case of $\SU(3)$, $(\lambda_1^\prime,\lambda_2^\prime)$
  lie in the shaded hexagonal region.
  Elements of the center of the group $z$, $z^2$, lie on the vertices
  of the hexagon with the identity at the origin.
  The six triangular regions correspond to permutations of
  $\lambda_1^\prime$, $\lambda_2^\prime$, $\lambda_3^\prime$.   \label{hexagon}}
\end{figure}
%
The $\SU(3)$ case is shown in Fig.~\ref{hexagon}.
Using $\lambda_j^\prime$, we define $U^t$ as:
\be
U^t = V \begin{pmatrix}
    \mathrm{e}^{i\lambda_1^\prime t} & & &\\
    & \mathrm{e}^{i\lambda_2^\prime t} & &\\
    & & \ddots & \\
    & & & \mathrm{e}^{i\lambda_N^\prime t}\end{pmatrix} V^\da \; .
\eq
This definition of $U^t$ fulfills the conditions listed above.
Similarly, we define the logarithm of $U$ as:
\be
\log U = i V \begin{pmatrix}
    \lambda_1^\prime & & &\\
    & \lambda_2^\prime & &\\
    & & \ddots & \\
    & & & \lambda_N^\prime\end{pmatrix} V^\da \; .
\eq
Thus, the norm defined in Eqn.~(\ref{sunorm}) is simply
\be
\fnorm{U} = \sqrt{\sum_j {\lambda_j^\prime}^2} \; .
\eq



\section{Lattice configurations}
\label{configurations}

Lattice configurations were generated using the Chroma
code~\cite{edwards_chroma_2005}.
For $N>3$ it was necessary to add partial pivoting~\cite{golub_matrix_1996}
to the Chroma routine that calculates matrix determinants.  We then
verified that Chroma-generated values of $\plaquette_\mathrm{avg}$ agree with
Teper's results~\cite{lucini_$mathrmsun$_2002} for $N=3,4$.

The lattice configurations were sampled after 10,000 thermalizing
steps, with 3000 steps between samples for results that use
multiple lattice configurations.  We used 4 over-relaxations steps
for every heat bath step.


\section{Newton's method for minimizing the lattice norm}
\label{newtonNorm}

Let us use Newton's method to find a gauge transform, Eqn.~(\ref{gauget}),
that minimizes the lattice norm squared $\sum_{\mu,x}\fnorm{U_\mu^\prime(x)}^2$
as a function of $B_a(x)$,
where $U_G(x) = \exp\left(i B_a(x)\, T_a\right)$.
The strategy employed here closely follows the strategy used to find
saddle-points of the action, Section~\ref{saddlePoint} and
Appendix~\ref{krylov}, with one important difference:  we are trying
to find a minimum, rather than a saddle-point.  The $n_G$ partial
derivatives
%
\be
   \left.\frac{\partial}{\partial_{B_b(y)}} \sum_{\mu,x}
   \fnorm{U_\mu^\prime(x)}^2\right|_{B_a(x)=0} =
   2 \Im \sum_\mu \Tr\left[T_b \log\left(U_\mu(y)\right)
             - \log\left(U_\mu(y-a\hat\mu)\right)T_b\right]
   \label{normGrad}
\eq
can be used to construct the gradient vector $\mathbf{q}$.
Likewise, one can calculate the $n_G\times n_G$ Hessian matrix
$H$ from
\be
   \left.\frac{\partial^2}{\partial_{B_b(y)}\partial_{B_c(z)}} \sum_{\mu,x}
         \fnorm{U_\mu^\prime(x)}^2\right|_{B_a(x)=0} \; .
\eq
which is nonzero when $y$ and $z$ are the same site or neighboring sites.
The full expression for $H$ is rather lengthy and
will not be given here.  Also, we construct a vector
$\mathbf{b} =  \left(B_1(x_1), \ldots \right)$ representing the
gauge transform.

As with the saddle point calculation, we define two
norms for the gauge transform:
a lattice-wide $\infty$-norm,
\be
\left\lVert U_G \right\rVert_{\mathrm{max}} =
     \max_{x} \fnorm{e^{i B_{a}(x) T_a}}
     = \max_{x} \sqrt{\frac{1}{2}\sum_a B_a^2(x)}
\eq
as well as a lattice-wide Euclidean norm, averaged over sites
\be
\left\lVert U_G\right\rVert_2 =
      \sqrt{\frac{1}{V_L} \sum_{x}
        \fnorm{e^{i B_{a}(x) T_a}}^2}
     = \sqrt{\frac{1}{2 V_L} \sum_{x, a} B_a^2(x)}
        \; .  \label{normshiftsize}
\eq

\begin{figure}
\includegraphics{normShiftPairs0}\\
\includegraphics{normShiftPairs1}
\caption{Scatter plot of eigenvalue-gradient pairs of $H^\prime$
  for a $6^3$ lattice, $\beta = 8.175$, $N=3$.
  The first plot is for a typical lattice configuration before any
  gauge fixing.  The second plot is for a lattice configuration
  after 3-axial gauge steps and 6 ``Landau steps'' have been
  applied.
  \label{normShiftPairs}}
\end{figure}

The lattice norm is invariant under global color rotations and
the gradient will be zero in those directions.
Thus, we introduce an $(N^2-1)\times n_G$ matrix $G$
representing $N^2-1$ infinitesmimal global color rotations.
We use the associated null-space operator $P_G$,
\be
G P_G^T = 0 \;, \quad P_G P_G^T = \mathbb{I} \; ,
\eq
to project $H$ and $\mathbf{q}$ onto
the space of non-global-color-rotation transforms:
\be
         H^\prime = P_G H P_G^T \quad \mbox{and} \quad
         \mathbf{q}^\prime = P_G \mathbf{q} \; .
\eq
%
Let $\mathbf{y}^\prime$ be a solution of the linear system
\be
   H^\prime \mathbf{y}^\prime = \mathbf{q}^\prime \label{normlinear2}
\eq
and define eigenvalues
$\left(\heigen_1, \heigen_2, \ldots\right)$ and
eigenvectors $V=\left(\mathbf{v}_1, \mathbf{v}_2, \ldots\right)$
of $H^\prime$ and
define $\overline{\mathbf{y}} = V \mathbf{y}^\prime$ and
$\overline{\mathbf{q}}  = V \mathbf{q}^\prime$.
 In any case where $\left|\heigen_i\right|\ll
\left|\overline{q}_i\right|$,
there will be a large gauge transform that will dominate the solution
to the linear system.
Likewise $\heigen_i<0$ corresponds to a local maximum of the function.
Thus, we impose a cutoffs $\Lambda_2$ and $\Lambda_\mathrm{max}$:
\begin{eqnarray}
    \sqrt{2} \left|\overline{q}_i\right|\left\lVert P_G^T \mathbf{v}_i\right\rVert_2
     &<& \heigen_i \, \Lambda_2 \label{normlambda2}\\
    \sqrt{2} \left|\overline{q}_i\right|
      \left\lVert P_G^T \mathbf{v}_i\right\rVert_\mathrm{max}
    &<& \heigen_i\, \Lambda_\mathrm{max} \; .
\end{eqnarray}
%
and set $\overline{y}_i=0$ whenever a cutoff is violated.
That is, we remove any eigenpairs to the left
of the slanted line shown in Fig.~\ref{normShiftPairs}.
The resulting gauge transform is
\be
      \mathbf{b} = -P_G^T \left. V^T
                    \overline{\mathbf{y}}\right|_{\Lambda_2,\Lambda_\mathrm{max}} \; .
\eq
%
(Numerically, we find that including a damping factor does not
improve convergence.)

Looking a Fig.~\ref{normShiftPairs}, we see that for a random
gauge choice, $H^\prime$ has many
negative eigenvalues and one should
expect that Newton's method will work very poorly.
However, after some intial minimization of the lattice
norm, we see that very few---if any---eigenvalues violate the cutoffs
and Newton's method should work quite well.


\section{Solving a constrained linear system}
\label{krylov}

For lattices of interest, the basis size $n_L$ is large enough
that direct solution is not possible and one must resort to Krylov
space methods (sparse matrix methods).
The Hessian matrix $H$ is sparse, with an $(N^2-1)\times(N^2-1)$
color block structure and $6D-5$ nonzero blocks per row/column.
The gauge transform matrix $G$ has the same color block structure
with $2 D$ nonzero blocks per row and $2$ nonzero blocks per column.

In order to preserve the sparsity of the matrices, we never
explicitly construct the space of non-transform shifts used in
Eqn.~(\ref{linear2}).  Instead, we apply the projection
operator $P_G$ to any Krylov space vector $\mathbf{v}$ using:
\be
    P_G \mathbf{v} = \mathbf{v} - G^T \mathbf{u} \; ,
\eq
where $\mathbf{u}$ is a solution of the linear system
\be
  G G^T \mathbf{u} = G \mathbf{v} \; .\label{gcg}
\eq
Eqn.~(\ref{gcg}) is solved using the conjugate gradient
algorithm, subroutine MINRES-QLP~\cite{choi_algorithm_2014}.
The matrix $G G^T$ is well-conditioned, so this typically converges
in less than 40 steps.

We use the Lanczos library nu-TRLan~\cite{yamazaki_adaptive_2010},
modified so that $P_G$ is applied to any Krylov space vector, to
find the smallest eigenpairs of $\left(P_G H\right)^2$.  From these, we find
the eigenvectors $\left\{\mathbf{r}_1, \mathbf{r}_2, \ldots\right\}$
associated with eigenpairs $\left(\heigen_i^2, \mathbf{r}_i\right)$
that violate the $\Lambda$ cutoffs,
%
\begin{eqnarray}
   \frac{\sqrt{2}\, \mathbf{r}_i \mathbf{\cdot} \mathbf{q}
   \left\lVert \mathbf{r}_i\right\rVert_2}{\left|\heigen_i\right| }
     &>& \Lambda_2  \quad\mbox{or} \\
   \frac{\sqrt{2}\, \mathbf{r}_i\mathbf{\cdot} \mathbf{q}
   \left\lVert \mathbf{r}_i\right\rVert_\mathrm{max}}{\left|\heigen_i\right| }
     &>& \Lambda_\mathrm{max} \; .
\end{eqnarray}
%
To ensure that we have found all the eigenpairs that violate the cutoffs,
we may need to calculate as many as 2000 to 4000 of the lowest eigenpairs
in the Lanzos routine.  As one moves along a trajectory towards the
saddle point, the gradient vector decreases in magnitude and fewer eigenpairs
exceed the $\Lambda$ cutoffs.

Finally, we solve the linear system $H \mathbf{y} = \mathbf{q}$
using the conjugate gradient algorithm,
subroutine MINRES-QLP~\cite{choi_algorithm_2014}, modified so that
the projection operators $P_G$ and 
$\mathbb{I}- \sum_{i} \mathbf{r}_i \otimes \mathbf{r}_i$
are applied to any Krylov space vector.

\section{Minimizing the action for a single link}

\label{linkmin}
To minimize the action for a single link $U\in\SU(N)$,
we want to find $U$ that minimizes
%
\be
S = -\Re\Tr U F \label{uf}
\eq
where $F$ is the sum of staples associated with $U$.
In general this involves minimizing a non-linear function of $N^2-1$
variables.  However, we can simplify the problem by working
in a basis where $F$ is diagonal.  Thus, we find the singular
value decomposition $F = V F^\prime W^\dag$ where $F^\prime$ is diagonal,
positive semi-definite and $V$ and $W$ are unitary.
If we define $U^\prime=W^\dag U V$, $U^\prime$ is unitary with some overall
phase $\theta/N$: $\det U^\prime = \det V W^\dag = e^{i \theta}$.
We make the {\em asatz} that $S$ is minimized by choosing $U^\prime$
to be diagonal:
\be
U^\prime = \begin{pmatrix}
  e^{i \lambda_1} & & & \\
  & \ddots & & \\
  & & e^{i \lambda_{N-1}} & \\
  & & & e^{i \lambda_N} \end{pmatrix}
\eq
where
\be
\lambda_N = \theta - \sum_{i=1}^{N-1} \lambda_i
\eq
Thus, one minimizes
\be
S = % -\Re\Tr\left(U^\prime F^\prime\right) =
    -\sum_{i=1}^N \cos\left(\lambda_i\right) F_{i,i}^\prime \label{diagmin}
\eq
with respect to $\left\{\lambda_1, \lambda_2, \ldots, \lambda_{N-1}\right\}$.
In practice, the minimum of (\ref{diagmin}) can be computed numerically
using standard library routines.
One can show that if $\left\{\lambda_1, \lambda_2, \ldots, \lambda_{N-1}\right\}$ is a minimum of
(\ref{diagmin}) then $U = W U^\prime V^\dag$ is indeed a global minimum
of Eqn.~(\ref{uf}) in the full $(N^2-1)$-dimensional space.

\bibliography{physics}  % Use physics.bib

\end{document}
