%\documentclass[twocolumn,eqsecnum,aps,]{revtex4-2}
\documentclass[preprint,aps,prd]{revtex4-2}
%\documentclass[eqsecnum,aps]{revtex}
%
%
%  Using Zotero to generate physics.bib file.
%
%  A number of sites set language=en which Zotero
%  passes on to BibTex, but LaTeX package babel doesn't know ``en.''
%  In Zotero, either remove the language field or set it to ``english''
%  before exporting to BibTex.
%\usepackage[english]{babel}
%
%
%  Ubuntu doesn't have the latest RevTex.
%  Download and install using:
%    sudo unzip revtex4-2-tds.zip -d /usr/share/texlive/texmf-dist/
%    sudo mktexlsr /usr/share/texlive/texmf-dist/
%
\bibliographystyle{apsrev4-2}
%
%
\usepackage{amsmath}
\usepackage{amsfonts}
\usepackage{graphicx}
%
\newcommand{\da}{\dagger}  % symbol for Hermitian conjugate dagger
\newcommand{\be}{\begin{equation}}
\newcommand{\eq}{\end{equation}}
\newcommand{\integer}{\mathbb{Z}}       % set of integers
\newcommand{\zentrum}{\mathcal{Z}}       % set of integers
\newcommand{\plaquette}{\Box}
\newcommand{\config}{\mathcal{U}}
\newcommand{\orelax}{\xi}
\newcommand{\heigen}{h}
\DeclareMathOperator{\SU}{SU}
\DeclareMathOperator{\Tr}{Tr}


\begin{document}
%\draft
\title{Saddle points of the Yang Mills action in 2+1 dimensions}


\author{Brett van de Sande}
%\affiliation{}
\noaffiliation

\begin{abstract}
  We find saddle points of $\SU(N)$ action in three space-time dimensions
  and see characteristics consistent with Witten's Master Field.
\end{abstract}

\pacs{Valid PACS appear here.
{\tt$\backslash$\string pacs\{\}} should always be input,
even if empty.}
\maketitle

%%%%%%%%%%%%%%%%%%%%%%%%%%%%%%%%%%%%%%%%%%%%%%%%%%%%%%%%%%%%%%%%%%%%%%
%%%%%%%%%%%%%%%%%%%%%%%%%%%%%%%%%%%%%%%%%%%%%%%%%%%%%%%%%%%%%%%%%%%%%%
%%%%%%%%%%%%%%%%%%%%%%%%%%%%%%%%%%%%%%%%%%%%%%%%%%%%%%%%%%%%%%%%%%%%%%
\baselineskip .2in

\section{Introduction}
\label{intro}

Since Wilson introduced his lattice action in 1974~\cite{wilson_confinement_1974},
lattice gauge theory has become the tool of choice for
non-perturbative studies of QCD.  Although numerical
calculations using Wilson's lattice have generated numerous
physical results, the nature of the QCD vacuum itself
has remained somewhat of a mystery.

% In 1982 't Hooft suggested that center vortices could
% explain confinement and subsequent numerical experiments
% using maximal abelian gauge have lent some evidence for this
% picture.  However, there has been little progress in taking
% this picture and creating a full description of the
% QCD vacuum state.  [Hugo Reinhardt's ugly paper is a
% heroic attempt in this direction.]

In 1979, Witten pointed out that QCD has some rather remarkable
properties in the $N \to \infty$ limit.  In particular,
expectation values of observables become dominated by a single
classical configuration of the gauge field, the
``master field''~\cite{witten_1/n_1980}.
Although this idea was compelling, there seemed to be no
way to figure out exactly what this Master Field could be.

To motivate our approach to studying the QCD vacuum, note the following:

\begin{enumerate}

\item Quarks are not important.  In the language of Lattice QCD,
 one says that the quenched approximation (no dynamical
 quarks) is a good approximation to the full theory.

\item The $N\to\infty$ limit of QCD is a good approximation to the
 physical case $N=3$.  This has been demonstrated in numerous
 perturbative and non-perturbative calculations~\cite{lucini_sun_2013}.
 In particular, Teper and collborators have shown this to be the case for
 the glueball spectrum in both 2+1~\cite{teper_$mathrmsun$_1998,lucini_$mathrmsun$_2002,athenodorou_sun_2017}
 and 3+1 dimensions~\cite{teper_large-n_2005}.

\item The behavior of $\SU(N)$ gauge theory in 3 space-time dimensions
  is remarkabley similar to the 4 dimensional theory~\cite{teper_$mathrmsun$_1998,lucini_$mathrmsun$_2002,teper_large-n_2005}.
  Although the scale is generated quite differently in each case ---
  in 4 dimensions, it is generated dynamically through quantum
  fluctuations while in 3 dimensions the coupling is dimensionful ---
  once a scale is generated, the theories behave in a very similar manner.

\item In Witten's Master Field picture, the master field is
  a semi-classical field.  That is, it is a saddle point
  of the QCD action.

\end{enumerate}

These facts suggest a research program for studying the QCD vacuum state:
one should look at saddle-points of the gauge field action
in 3 spacetime dimensions, looking for a description that emerges in
the $N\to \infty$ limit.  Hopefully, this description
can then be generalized to the 4 dimensional theory.

If we take the Master Field picture seriously, then a typical gauge
field configuration in a Monte Carlo simulation should contain
this master field and---to some approximation---be able to
produce the full spectrum of long-range observables. This
approximation should become increasingly precise as $N\to\infty$.

I pursue the first step of this program in the following.
We will start with conventional 2+1 dimension lattice gauge
theory field configurations for various values of $N$ and
calculate the nearest associated saddle-point configurations.
These saddle-point field configurations should provide
a good starting-point for understanding the 2+1 dimension
vacuum in the $N\to\infty$ limit.

\section{The Wilson action}

The purpose of the section is to set some notation.
We introduce a $D$-dimensional spacetime lattice with lattice
spacing $a$ and sites labeled by $x$, with a total of $V_L$ sites.
We will focus on cubic lattices with $L$ sites in each direction,
$V_L = L^D$.
Define $U_\mu(x) \in \SU(N)$ to be the link going from site $x$ to
site $x+a \hat{\mu}$.
The lattice has $n_L=V_L D \left(N^2-1\right)$ degrees of freedom.
Let lattice configuration
$\config=\left\{U_1(x_1),\ldots\right\}$
be a particular set of link fields that cover the lattice.

The associated Wilson action is
%
\be
S = \beta \sum_{x,\, \mu<\nu} \left(1-\frac{1}{N} \Re\, \plaquette_{\mu,\nu}(x)\right) \label{action}
\eq
where
\be
\beta=\frac{2 N}{a g^2}
\eq
so that $S$ becomes the usual Yang-Mills action in the $a\to 0$ limit.
In $D=3$ dimensions, the coupling $g$ has dimensions of
$\left(\mbox{mass}\right)^{1/2}$, consistent with $\beta$ being dimensionless.
Also, $\plaquette_{\mu,\nu}(x)$ is a plaquette lying in the $\mu\nu$-plane:
\be
\plaquette_{\mu,\nu}(x) = \Tr U_\mu(x) U_\nu(x+a \hat{\mu})
U_\mu^\da(x+a\hat{\nu}) U_\nu^\da(x) \; .
\eq
It is sometimes convenient to use the average plaquette,
\be
      \plaquette_\mathrm{avg} = \frac{2}{n_L (D-1) N}
             \sum_{x,\,\mu<\nu} \Re\,\plaquette_{\mu,\nu}(x) 
\eq
with normalization such that $\plaquette_\mathrm{avg}=1$
for the bare vacuum configuration $U_\mu(x) = \mathbb{I}$.

The Wilson action is invariant under $\SU(N)$ gauge transforms:
%
\be
    U_\mu(x) \to U_G(x) U_\mu(x) U_G^\da(x+a \hat{\mu})
         \; , \quad U_G(x) \in \SU(N) \; .
\eq
%
The lattice has a total of $n_G = V_L\left(N^2-1\right)$ gauge symmetries.

\section{Norms}

It is useful to define a number of norms.  We define the norm
for a link,
%
\be
   \left\lVert U_\mu(x) \right\rVert_2 =
   \sqrt{\Tr B B^\dagger} \quad \mbox{where}
   \quad B = \log\left(U_\mu(x)\right)
   \label{sunorm}
\eq
and the matrix logarithm is defined in Appendix~\ref{roots}.
Note that the norm reaches its maximum value for elements
of the center $\zentrum$ of $\SU(N)$: 
\be
    \left\lVert z \right\rVert_2 = 2\pi \sqrt{\frac{N-1}{N}} \; ,
     \quad z\in \zentrum \; .
\eq
The norm for a lattice configuration is:
\be
   \left\lVert \config \right\rVert_2 =
   \sqrt{\frac{1}{V_L D} \sum_{\mu,x} \left\lVert U_\mu(x) \right\rVert_2^2}
   \; . \label{latnorm2}
\eq
%
%and the distance between two lattice configurations,
%\be
%   \left\langle \config, \config^\prime \right\rangle_2 =
%   \sqrt{\frac{1}{V_L D} \sum_{\mu,x} \left\lVert U_\mu^\dagger(x)
%     U_\mu^\prime(x) \right\rVert_2^2} \; .
%\eq
%The distance metric should not be taken too seriously, 
%since it is not gauge invarient.
%
We can also define a link norm modulo the center $\zentrum$,
\be
\left\lVert U_\mu(x) \right\rVert_\zentrum =
    \min_{z\in \zentrum} \left\lVert z U_\mu(x) \right\rVert_2 \; .
\eq
That is, $\left\lVert U_\mu(x) \right\rVert_\zentrum$ is the distance
of $U_\mu(x)$ from the nearest element of $\zentrum$.
Analogous to Eqn.~(\ref{latnorm2}), we can
use $\left\lVert U_\mu(x) \right\rVert_\zentrum$ to define
the associated lattice norm $\left\lVert \config \right\rVert_\zentrum$.

\section{Observables}

A number of observables and gauge choices will be used to
monitor the behavior of the gauge fields when moving from
the original Euclidean lattice configuration toward
the saddle-point.

\subsection{Loop Operators}

As we move toward the saddle-point of a lattice configuration,
we expect that the perturbative (quantum) fluctuations of
the gauge field to change greatly.  The standard strategies for
improving lattice operators~\cite{teper_$mathrmsun$_1998,lucini_glueballs_2004}
are designed precisely to ameliorate the effects of these fluctuations.
It is not clear how they should behave as these flucuations
are removed.  In the current work, we will stick to
the bare (unimproved) operators.
In particular, we will use Polyakov loop correlators to measure
string tension and use Wilson loops and Wilson loop correlators to
study properties of the fields.

Define a Polyakov loop that winds once around the lattice
in the direction $\mu$ and goes through site $x$:
\be
         \phi_\mu(x)= \prod_{\rho=1}^L U_\mu(x+a \hat{\mu} \rho)
\eq
Define the correlator:
\be
\Phi_\mu(r) = \frac{1}{C_\phi} \sum_{x, x_\mu=1} \sum_{y, y_\mu=1}
           \delta_{r,|x-y|}
           \langle 0 | \frac{1}{N} \Tr\left(\phi_\mu^\dagger(y)\right)
           \frac{1}{N} \Tr\left(\phi_\mu(x)\right) |0\rangle
           \label{pcorr}
\eq
%
where the normalization constant $C_\Phi$ is chosen such that
$\Phi_\mu(r)=1$ for the bare vacuum.  To obtain the string tension,
we fit the correlator to an exponential plus a constant term,
including the bosonic string correction in the $\hat{\mu}$
direction:
\be
P_\mu(r) = c_0 \sum_{\mbox{images}}
      \exp\left(\sigma a^2 r L_\mu\sqrt{1-\frac{\pi}{3 a^2 \sigma L_\mu^2}}\right)
      + c_1 \; .  \label{stringmodel}
\eq    
Since the lattices we use are relatively small, we include contributions
from fields wrapping around the lattice in the plane transverse to
$\hat{\mu}$.  Thus we sum over the two closest images of the Polyakokv
loop in each transverse direction.  The constant $c_1$ should,
in principle, not be necessary but when measuring the string tension
for a single lattice configuration, we find large constant
fluctuations will otherwise interfere with the fit.
When finding the string tension, we fit $\Phi_\mu(r)$ to
Eqn.~(\ref{stringmodel}) for each $\mu$.

\begin{figure}
  \includegraphics{polycorr16}
  \caption{Polyakov loop correlator vs.\ loop area together with
    a fit to Eqn.~(\ref{stringmodel}) for $L_\mu r>40$.  The solid line
    represents the case where the separation of $x$
    and $y$ is along a lattice axis and the dashed line
    represents the case where the separation is in
    a diagonal direction.  The data is from 100 lattice
    configurations. \label{pcorr16}}
\end{figure}

For precise measurement the string tension, the above process is
not ideal since the correlator calculation re-uses the same link
fields for various values of $r$ and $\mu$; the $\Phi_\mu(r)$ are not
statstically independent.
However, this does allow us to monitor the behavior of the correlator at
various distance scales.  An example for a $16^3$ lattice is shown
in Fig.~\ref{pcorr16}.  The lattice configurations were generated with the
Chroma library~\cite{edwards_chroma_2005};
see Appendix~\ref{configurations} for details.
The string tension $\sigma a^2 = 0.0266(34)$
is somewhat larger then Lucini and Teper's value for $\beta=28$,
$\sigma a^2 = 0.016192(54)$ \cite{lucini_$mathrmsun$_2002,athenodorou_sun_2017}
and is likely due to the Coulomb force. {\bf fix this!}
A similar calculation for a $L^3=20^3$ lattice yields
$\sigma a^2 = 0.0136(23)$, closer to Lucini and Teper's value.
The fit has $\chi^2=8.2$ for 37 degrees of freedom, reflecting
the fact that the correlators are not statistically independent.

Similarly, we define a Wilson loop operator using a product of
link fields lying on the perimeter a rectangle having the given
dimensions:
\be
       W_{m\times n} = \frac{1}{C_W} \sum_{\mu \ne \nu, x}
         \prod_{\mbox{$m\times n$ rectangle}}
         U_\mu(x) U_\mu(x+a \hat{\mu}) \,\cdots\, U_\nu^\dagger (x)
\eq
with normalization $C_W$ chosen such that $W_{m\times n} = \mathbb{I}$
for the bare vacuum.

Also, the quantity that is actually measured can be generalized.
For some loop operator $\mathcal{O}\in\SU(N)$, one usually
measures the expectation value of the trace,
$\langle 0 | \frac{1}{N} \Tr \mathcal{O} |0\rangle$, or for $k$-strings,
one can measure expectation values of higher order operators
like $\Tr\left( \mathcal{O}^k\right)$.
However, this is not the most general gauge-invarient observable.
The most general physical observable is the distribution of the
eigenvalues of $\mathcal{O}$.  If $e^{i\lambda_1}$,\ldots, $e^{i\lambda_N}$
are the eigevalues of $\mathcal{O}$, we can order $\lambda_j$ in
some fashion $\lambda_1\ge \lambda_2 \ge \ldots$ and demand that
$\lambda_j-\lambda_k\le 2\pi$ (see Appendix~\ref{roots}).
In that case, physically
distinct values lie on an ($N-1$)-simplex with the elements of the
center of $\SU(N)$ on the vertices.

\begin{figure}
  \includegraphics{polydist1}
  \includegraphics{polydistr}
  \caption{Distribution for the Wilson loop operator
    $W_{5\times 5}$ for $\SU(3)$, $\beta=28$, $L^3=16^2$,
    averaging over 100 lattice configurations.
    Also shown is the distribution for the high temperature
    limit.  The distribution is
    symmetric under charge conjugation,
    % $2 \lambda_1 +\lambda_2 \to 2 \lambda_1 +\lambda_2 $,
    $\lambda_2 \to -\lambda_2$.
   \label{polydist1}}
\end{figure}
\begin{figure}
  \includegraphics{polydist2}
  \caption{Distribution for $\Tr W_{5\times 5}/N$  for
    the data shown in Fig.~\ref{polydist1}.  
    The distribution is symmetric under complex conjugation,
    $\mathcal{O}\to\mathcal{O}^\dagger$.
    The distribution lies inside the blue lines representing
    the edges of the 2-simplex in Fig.~\ref{polydist1}.
    \label{polydist2}}
\end{figure}

By way of example, we show the distribution of the
$5\times5$ Wilson loop operator for $\SU(3)$ in Fig.~\ref{polydist1}.
For comparision, we also show the distribution for
the high temperature limit (random $\SU(N)$ matrices).
Notice that the shape of the two distributions appear to
be quite similar, except that the $W_{5\times5}$ distribution
is shifted toward $\mathbb{I}$, reflecting the fact that
$\langle\Tr W_{5\times5}\rangle >0$.

We can relate the distribution in $\lambda_j$-space to $\Tr\mathcal{O}$.
For $\SU(2)$, $\Tr \mathcal{O}$ is real and for $\SU(3)$, $\Tr \mathcal{O}$
is complex.  In both cases, the map from the $\lambda_j$-space to
$\Tr \mathcal{O}$ is injective.  Thus, one can just as well look at the
distribution of $\Tr \mathcal{O}$; see Fig.~\ref{polydist2}.
However, for $N>3$, information is lost when taking the trace
of $\mathcal{O}$.

\subsection{Landau gauge}

The choice of Landau gauge $\partial_\mu A^\mu = 0$ is equivalent to
minimizing $\Tr A_\mu A^\mu$ over all space.  The fact that
$\partial_\mu A^\mu = 0$ has multiple solutions (the Gribov problem) is
equivalent to the statement that $\Tr A_\mu A^\mu$, integrated over space,
has multiple local minima.~\cite{maas_more_2009}.
We are interested here in Landau gauge mainly
because it allows us, if $\Tr A_\mu A^\mu$ is small enough, to connect the
link fields to the continuum degrees of freedom.

However, finding this minimum can be a bit tricky.
We will start with local minimization.
For some site $x$, we want to find a gauge transform $U_G$ on $x$
that minimizes $\left\lVert U_\mu(x) \right\rVert_2$ for the $2D$
links connected to $x$.  Taylor expanding in the link fields,
we find that, to lowest order in the fields, a gauge transform
that minimizes this norm:
%
\be
          U_G(x) = e^{- B}
\eq
where
\be
   B = \frac{\orelax}{2 D} \sum_\mu \log\left(U_\mu(x)\right) -
   \log(U_\mu\left(x-a \hat{\mu})\right)  \; . \label{landau}
\eq
At each site of the lattice, one can calculate $U_G(x)$ and apply it,
moving through the lattice sites in a checkerboard fashion,
first all the even sites and then all the odd sites.
This constitutes one minimization step, a ``Landau step,'' in a
sucessive over-relaxation (SOR) procedure.
The relaxation factor $\orelax$ can be adjusted to speed numerical
convergence.
In practice, we find the fastest convergence for $\orelax \approx 1.7$; however,
it can become unstable for slightly higher $\orelax$.
We generally use $\orelax=1.5$ as a compromise between speed and stability.

\subsection{Maximal center gauge}

Although maximal center gauge~\cite{del_debbio_center_1997}
has been an important topic in the
literature~\cite{greensite_confinement_2003},
we will not consider it separately from Landau gauge.
This section explains why and introduces our best algorithm
for finding Landau gauge.

\begin{table}
  \caption{Local search for maximal center gauge,
    showing successive applications of ``center step'' to
    an $\SU(3)$ lattice configuration ($L^3=16^3$, $N=3$, $\beta=28$).
    The last three columns show the number of links closest
    to each element of the center, $z^k \in \zentrum$.
    \label{center1}}
  \begin{tabular}{c|c|c|ccc}
    iteration $i$ & operation & $\left\lVert \config_i \right\rVert_\zentrum$
     & $z^0=\mathbb{I}$ & $z^1$ & $z^2$ \\
    \hline
    0 & original config & 2.377 & 4101 & 4121 & 4066\\
    1 & center step     & 1.811 & 4090 & 4145 & 4053\\
    2 & center step     & 1.625 & 4085 & 4147 & 4056\\
    3 & center step     & 1.568 & 4085 & 4150 & 4053\\
    \vdots & \vdots & \vdots & \multicolumn{3}{c}{\vdots}\\
    11 & center step    & 1.503 & 4090 & 4144 & 4054\\
    12 & center step    & 1.501 & 4090 & 4144 & 4054\\
  \end{tabular}
\end{table}

\begin{table}
  \caption{Successive applications of $\zentrum$-axial gauge
    and ``center step'' to the lattice configuration
    used in Table~\ref{center1}.
    The last three columns show the number of links closest
    to $z^k \in \zentrum$.
    \label{center2}}
  \begin{tabular}{c|c|c|ccc}
    iteration $i$ & operation & $\left\lVert \config_i \right\rVert_\zentrum$
     & $z^0=\mathbb{I}$ & $z^1$ & $z^2$ \\
    \hline
    0 & original config              & 2.377  & 4101 & 4121 & 4066\\
    1 & $\zentrum$-axial gauge $\mu=1$& 1.950  & 4140 & 4053 & 4095\\
    2 & $\zentrum$-axial gauge $\mu=2$& 1.664  & 4189 & 4024 & 4075\\
    3 & center step                  & 1.268  & 4191 & 4023 & 4074\\
    4 & $\zentrum$-axial gauge $\mu=3$& 1.553  & 4116 & 4073 & 4099\\
    5 & center step                  & 1.120  & 4121 & 4076 & 4091\\
    6 & center step                  & 0.9565 & 4121 & 4080 & 4087\\
    7 & center step                  & 0.9022 & 4115 & 4082 & 4091\\
    \vdots & \vdots & \vdots & \multicolumn{3}{c}{\vdots}\\
    19 & center step                 & 0.8636 & 4117 & 4082 & 4089\\
    20 & center step                 & 0.8632 & 4117 & 4082 & 4089\\
  \end{tabular}
  \end{table}

\begin{table}
  \caption{Successive applications of axial gauge
    and ``center step'' to the lattice configuration
    used in Table~\ref{center1}.
    The last three columns show the number of links closest
    to $z^k \in \zentrum$.
    \label{center3}}
  \begin{tabular}{c|c|c|ccc}
    iteration $i$ & operation & $\left\lVert \config_i \right\rVert_\zentrum$
     & $z^0=\mathbb{I}$ & $z^1$ & $z^2$ \\
    \hline
    0 & original config    & 2.377  & 4101 & 4121 & 4066\\
    1 & axial gauge $\mu=1$& 1.945  & 6826 & 2805 & 2657\\
    2 & axial gauge $\mu=2$& 1.626  & 9668 & 1450 & 1170\\
    3 & center step        & 1.244  & 9674 & 1450 & 1164\\
    4 & axial gauge $\mu=3$& 1.455  & 11369 & 386 & 533\\
    5 & center step        & 1.035  & 11409 & 363 & 516\\
    6 & center step        & 0.8543 & 11455 & 340 & 493\\
    7 & center step        & 0.7887 & 11471 & 334 & 483\\
    \vdots & \vdots & \vdots & \multicolumn{3}{c}{\vdots}\\
    19 & center step       & 0.7219 & 11498 & 317 & 473\\
    20 & center step       & 0.7214 & 11498 & 317 & 473\\
  \end{tabular}
  \end{table}

\begin{table}
  \caption{Successive applications of axial gauge,
    ``Landau step,'' and ``center step'' to the lattice configuration
    used in Table~\ref{center1}.
    The last three columns show the number of links closest
    to $z^k \in \zentrum$.
    \label{center4}}
  \begin{tabular}{c|c|c|ccc}
    iteration $i$ & operation & $\left\lVert \config_i \right\rVert_\zentrum$
     & $z^0=\mathbb{I}$ & $z^1$ & $z^2$ \\
    \hline
    0 & original config    & 2.377  & 4101 & 4121 & 4066\\
    1 & axial gauge $\mu=1$& 1.945  & 6826 & 2805 & 2657\\
    2 & axial gauge $\mu=2$& 1.626  & 9668 & 1450 & 1170\\
    3 & center step        & 1.244  & 9674 & 1450 & 1164\\
    4 & axial gauge $\mu=3$& 1.455  & 11369 & 386 & 533\\
    5 & Landau step        & 1.018  & 12271 &   8 &   9\\
    6 & center step        & 0.7231 & 12268 &   9 &  11\\
    7 & Landau step        & 0.6009 & 12287 &   0 &   1\\
    8 & center step        & 0.5426 & 12286 &   1 &   1\\
    9 & center step        & 0.5128 & 12286 &   1 &   1\\
    10 & center step       & 0.4952 & 12286 &   1 &   1\\
    11 & center step       & 0.4835 & 12286 &   1 &   1\\
    12 & center step       & 0.4751 & 12287 &   1 &   0\\
  \end{tabular}
  \end{table}

We can use a procedure similar the one introduced
in the previous section to minimize norm of the link fields
relative to the center $\zentrum$ of the group.
We can do this by computing the logarithms in Eqn.~(\ref{landau})
using
\be
\log\left(U_\mu(x)\right) \to \log\left(z^k U_\mu(x)\right)
\, , \;\mbox{
  where $z^k\in\zentrum$ minimizes}\; \left\lVert z^k U_\mu(x) \right\rVert_2 \;.
\eq
One can then apply the associated gauge transform to each site of
the lattice as described above, a ``center step.''
If we apply some number of ``center steps'' to a lattice
configuration, we find quick convergence to some local
minimum of $\left\lVert \config \right\rVert_\zentrum$ with
link fields distributed fairly evenly across $\zentrum$.
See Table~\ref{center1} for an example.

However, this procedure fails to remove long-range field fluctuations
associated with the random choice of gauge.  We can remove
such fluctuations by another gauge choice:  axial gauge
$\partial_\mu A_\mu = 0$ (no sum on $\mu$).
That is, we choose a gauge such that $U_\mu(x)$ is replaced
by the $L_\mu^\mathrm{th}$ root of the associated Polyakov loop,
$U_\mu(x) \to \phi_\mu(x)^{1/L_\mu}$ (the root is defined in
Appendix~\ref{roots}).
% While not a complete gauge fixing, this gauge choice does
% remove most long-scale fluctuations caused by the
% gauge choice itself.
This gauge choice minimizes $\left\lVert U_\mu(x) \right\rVert_2$
on the link fields in $\phi_\mu(x)$.  One might
object that this is not appropriate for maximal center gauge
since it preferetially minimizes the link fields
with respect to $\mathbb{I}$ and not some other element of $\zentrum$.
This leads us to a variation which we will call ``$\zentrum$-axial gauge''
where we minimize $\left\lVert U_\mu(x) \right\rVert_\zentrum$
on the link fields in $\phi_\mu(x)$.  In Table~\ref{center2},
we apply this gauge transform in each direction before applying
some number of local minimizations.  One can see that this
results in a substantially improved lattice norm relative to
Table~\ref{center1}.  We also see that the link fields
remain relatively evenly distributed over the elements of $\zentrum$.

But is $\zentrum$-axial gauge actually an improvement over
conventional axial gauge?
In Table~\ref{center3}, we apply axial gauge in each
direction before applying some number of ``center steps.''
We see that this produces a lattice norm that is, in fact,
substantially lower than before.  We also can see, since
axial gauge preferentially minimizes $U_\mu(x)$ with
respect to $\mathbb{I}$, substantially more links near
the identity element.

Can we do better? Let us include some number of ``Landau steps''
in the procedure; see Table~\ref{center4}.  (The strategy illustrated
here was the end result of a an extensive monte carlo study that
included a number of other gauge fixing strategies, as well.)
We see a substantially reduced lattice norm, relative to
Table~\ref{center3}.  We also see that virtually all the
links are closest to identity element.  Thus, this gauge
choice also minimizes $\left\lVert \config_k \right\rVert_2$,
which corresponds to Landau gauge.
We won't give details here, but similar calculations for
$\SU(2)$ and $\SU(4)$ produce comparable results.
In conclusion, we find evidence that, for $D=3$,
``absolute'' maximal center gauge is identical to ``absolute''
Landau gauge (using the language of Ref.~\cite{maas_more_2009});
Table~\ref{center4} gives
an effective strategy for finding the associated minimum.

This analysis casts doubt on the various lattice studies involving
maximal center gauge and center projections~\cite{
del_debbio_center_1997,del_debbio_detection_1998,de_forcrand_relevance_1999}.
Presumably, in these studies, relatively large local minima of
$\left\lVert \config_k \right\rVert_\zentrum$ were being investigated
rather than configurations near the global minimum.


\section{Shifts of the gauge fields}

\begin{figure}
  \[
  \begin{array}{l}
    \mbox{} \hspace{0.2in} x \hspace{0.3in} \sqrt{U_\mu(x)}
    \hspace{0.25in} e^{i w_{a}(\mu, x) T_a}
    \hspace{0.25in} \sqrt{U_\mu(x)} \hspace{0.3in} x+a\hat{\mu} \\[-0.3in]
    %
    %  Graphic generated in Mathematica file ``gauge.nb''
    %
  \includegraphics{link}
  \end{array}
  \]
  \caption{One can think of a shift as applying a color rotation
    to the middle of a lattice link. \label{shift}}
\end{figure}

In order to find a saddle point, it is convenient to modify
the link fields $U_\mu(x)$ in a manner that maintains lattice symmetries.
We define a ``shift'' for each link as
\be
  U_\mu(x) \to \sqrt{U_\mu(x)}\, e^{i w_{a}(\mu, x) T_a}\, \sqrt{U_\mu(x)} \; ,
    \label{shifts}
\eq
where $T_a$ are generators of $\SU(N)$, with normalization
$\delta_{a,b} = 2 \Tr(T_a T_b)$ and the square root is defined
in Appendix~\ref{roots}.
We can then take partial derivatives of $N S/\beta$ with
respect to the $n_L$ shifts $\mathbf{w} = \left(w_a(\mu, x), \ldots\right)$
to obtain the gradient vector $\mathbf{q}$ and Hessian matrix
$H$.  The gradient and Hessian can be used to estimate the
position of the nearest saddle point.

Let us consider the gradient for the link $U_\mu(x)$.
The gradient will involve the $2 (D-1)$ plaquettes that include that link.
Let $F_\mu(x)$ be the associated sum of staples; then
\be
   \frac{N}{\beta} \frac{\partial S}{\partial w_a(\mu, x)} =
   \Im\Tr\left(F_\mu(x) \sqrt{U_\mu(x)} T_a \sqrt{U_\mu(x)}\right) \; .
   \label{grad}
\eq
We can use Eqn.~(\ref{grad}) to construct the gradient vector $\mathbf{q}$.
Likewise, one can calculate the $n_L\times n_L$ Hessian matrix
$H$ from
\be
      \frac{N}{\beta} \frac{\partial^2 S}{\partial w_a(\mu, x)\, \partial w_b(\nu, y)}
\eq
which is nonzero when there is a plaquette which contains both
$U_\mu(x)$ and $U_\nu(y)$.
The full calculation of $H$ is rather lengthy and we will not bore the reader
with the details.

Some shifts $\mathbf{w}$ correspond to gauge transforms.
As we shall see, such shifts are problematic for any saddle
point search, since the action is constant in those directions.
To address this issue, we determine the set of shifts that are
equivalent to infinitesimal gauge transforms.  Consider a gauge transform
at site $x$; its action on link $U_\mu(x)$ is:
\be
U_\mu(x) \to e^{i C_a(x) T_a} U_\mu(x) = U_\mu(x) + i C_a(x) T_a U_\mu(x) +
       \mathrm{O}\!\left(C_a^2\right)
\eq
But this is equal to some infinitesimal shift $w_a(\mu, x)$ on
the same link $U_\mu(x)$:
\be
U_\mu(x) \to U_\mu(x) + i w_a(\mu,x) \sqrt{U_\mu(x)}T_a \sqrt{U_\mu(x)} +
       \mathrm{O}\!\left(w_a^2\right)
\eq
%
Equating the two and taking the trace:
\be
w_a(\mu,x) = 2 C_b(x) \Tr\left(\sqrt{U_\mu(x)} T_a
                     \sqrt{U_\mu^\da(x)} T_b\right) \; . \label{gs}
\eq
Using Eqn.~(\ref{gs}), one can construct an $n_G \times n_L$ matrix
$G$ that relates shifts to infinitesimal gauge transforms
at each lattice site.
Since the action is invariant under gauge transforms, $G \mathbf{q} = 0$.
However, since the Hessian represents the quadratic term, $G H G^T \neq 0$.

Since we are using the Hessian, we are approximating
the action $S$ as being quadratic in $\mathbf{w}$.  If $\mathbf{w}$
is too large, the quadratic approximation is no longer valid.
Thus, we need to monitor the size of the shifts.
Using Eqn.~(\ref{sunorm}), we will define two norms for the shifts:
a lattice-wide $\infty$-norm, taking the ordinary norm at each link
\be
\left\lVert \mathbf{w}\right\rVert_{\mathrm{max}} =
     \max_{x,\mu} \left\lVert e^{i w_{a}(\mu, x) T_a} \right\rVert_2
     = \max_{x,\mu} \sqrt{\frac{1}{2}\sum_a w_a^2(\mu, x)}
\eq
as well as an ordinary Euclidean norm, averaged over links
\be
\left\lVert \mathbf{w}\right\rVert_2 =
      \sqrt{\frac{1}{V_L D} \sum_{x, \mu}
        \left\lVert e^{i w_{a}(\mu, x) T_a} \right\rVert_2^2}
     = \sqrt{\frac{1}{2 V_L D} \sum_{x, \mu, a} w_a^2(\mu, x)}
        \; .  \label{shiftsize}
\eq

\section{Calculating the saddle point}
\label{saddle}

\begin{figure}
\includegraphics{eigenAll}
\caption{Histogram of the eigenvalues of $H$ for a $6^3$ lattice,
  $\beta = 8.175$, $N=3$.  We see a peak around zero due to the
  $\SU(3)$ gauge symmetry.
  \label{eigenAll}}
\end{figure}

In principle, to find the nearest saddle point, one would
find $H$ and $\mathbf{q}$, solve the linear system
\be
    H \mathbf{y} = \mathbf{q} \label{linear1}
\eq
and shift the gauge fields using $\mathbf{w} = -\mathbf{y}$.
Since the action $S$ is not quadratic in $\mathbf{w}$, one
would then iterate this process to find the actual saddle point.

However, in practice, this procedure does not work.  To see why, let us
look at the eigenvalues of $H$.  As seen in Figure~\ref{eigenAll},
we see that the eigenvalues are strongly peaked around zero,
due to the gauge symmetry of the theory.  Thus, the linear
system (\ref{linear1}) will be nearly singular and
the solution will be numerically unstable.  We must be more careful.

To address this issue, we remove directions in shift-space corresponding
to infinitesimal gauge transforms.  To do this, we start
by calculating the null-space $P_G$ of the gauge transform matrix $G$,
\be
G P_G^T = 0
\eq
where $P_G P_G^T = \mathbb{I}$.  Then we project $H$ and $\mathbf{q}$ onto
the space of non-gauge-transform shifts:
\be
         H^\prime = P_G H P_G^T \quad \mbox{and} \quad
         \mathbf{q}^\prime = P_G \mathbf{q} \; .
\eq
Then we can solve the linear system
\be
   H^\prime \mathbf{y}^\prime = \mathbf{q}^\prime \label{linear2}
\eq
and set $\mathbf{w} = - P_G^T \mathbf{y}^\prime$.  However, this is
still insufficient to yield a well-behaved linear system.

\begin{figure}
\includegraphics{shiftPairs}
\caption{Scatter plot of eigenvalue-gradient pairs of $H^\prime$
  for a $6^3$ lattice, $\beta = 8.175$, $N=3$.  The pairs above the
  v-shaped wedge correspond to large shifts and will dominate
  the solution to the linear system, Eqn.~(\ref{linear2}).
  \label{shiftPairs}}
\end{figure}

To better understand this issue, let us find the eigenvalues
$\left(\heigen_1, \heigen_2, \ldots\right)$ and
eigenvectors $V=\left(\mathbf{v}_1, \mathbf{v}_2, \ldots\right)$
of $H^\prime$ and express Eqn.~(\ref{linear2}) in the eigenspace of $H^\prime$:
\be
\begin{pmatrix}
    \heigen_1 & & \\
    & \heigen_2 & \\
    & & \ddots  \end{pmatrix} \overline{\mathbf{y}} =
  \overline{\mathbf{q}} \label{linear3}
\eq
where $\overline{\mathbf{y}} = V \mathbf{y}^\prime$ and
$\overline{\mathbf{q}}  = V \mathbf{q}^\prime$.
The solution is simply
\be
    \overline{y}_i = \frac{\overline{q}_i}{\heigen_i} \; .
\eq
In any case where $\left|\heigen_i\right|\ll
\left|\overline{q}_i\right|$,
there will be a large shift that will dominate the solution
to the linear system.
Thus, we impose a cutoff $\Lambda_2$:
\be
    \frac{\sqrt{2} \left|\overline{q}_i\right|\left\lVert P_G^T \mathbf{v}_i\right\rVert_2}{\left|\heigen_i\right|}
     < \Lambda_2 \label{lambda2}
\eq
%
and set $\overline{y}_i=0$ whenever the cutoff is violated.
That is, we remove any eigenpairs inside the v-shaped
region shown in Fig.~\ref{shiftPairs}.  For additional safety, we
we also impose a maximum value cutoff
\be
    \frac{\sqrt{2} \left|\overline{q}_i\right|
      \left\lVert P_G^T \mathbf{v}_i\right\rVert_\mathrm{max}}
    {\left|\heigen_i\right|}
    < \Lambda_\mathrm{max} \; .
\eq
Thus,
\be
      \mathbf{y} = P_G^T \left. V^T
                    \overline{\mathbf{y}}\right|_{\Lambda_2,\Lambda_\mathrm{max}} \; .
\eq

Finally, since the gauge symmetry of the action is non-linear,
we do not want to shift the fields by too much before recalculating
the matrix $G$.  In addition, we want to ensure the shift is small
enough on each link to be consistant with the quadratic expansion.
Thus, it is useful to add a damping factor $\gamma$ and
further rescale by $\Lambda_s$ when the shift would
otherwise be large:
\be
  \mathbf{w} = - \gamma \mathbf{y} \min\left(1, \frac{\Lambda_s}{
    \sqrt{2} \left\lVert \mathbf{y}\right\rVert_\mathrm{max}}\right) \; .
\eq
Typically, $\Lambda_s = 1$ or $2$ is used.

Armed with these three tools:  removing infinitesimal-gauge-transform shifts,
imposing the $\Lambda$ cutoffs, and using the damping factor $\gamma$,
we can apply the shift iteratively and find saddle-point configurations
that are of physical interest.

\section{Trajectories}

\begin{figure}
\includegraphics{trajectory3}
\caption{Several trajectories are shown for a $16^3$ lattice configuration,
  $N=3$, $\beta=28$.
  The starting configuration $\config_0$ was generated using standard Euclidean
  Lattice Monte Carlo;
  {\bf 2b} was generated with $\gamma=0.1$, $\Lambda_\mathrm{max} = 2$,
  and $\Lambda_2=0.04434$;
  {\bf 2c} was generated with $\gamma=0.5$, $\Lambda_\mathrm{max} = 2$,
  and $\Lambda_2=0.02217$;
  {\bf 2d} was generated with $\gamma=0.1$, $\Lambda_\mathrm{max} = 2$,
  and $\Lambda_2=0.02217$.
  For {\bf lc}, $\gamma=0.1$, off-diagonal blocks of $H$ were dropped,
  and a shift
  was applied to a link when $\left\lVert\mathbf{w}\right\rVert<0.327$.
  \label{trajectory3}}
\end{figure}
  
We will apply our procedure to $\SU(3)$, $16^3$ and $20^3$ lattice
configurations, as well as $\SU(4)$, $16^3$ lattice
configurations.  Details of the numerical calculation
are discussed in Appendices~\ref{krylov} and \ref{configurations}.

By applying the procedure discussed in Section~\ref{saddle} iteratively,
we produce a trajectory of lattice configurations $\config_0$,
$\config_1$, \ldots in configuration space.  For lattice configuration $\config_i$, we define
$\Delta_i$ to be the total distance traveled through configuration
space
\be
\Delta_i = \sum_{\rho=1}^i \left\lVert \mathbf{w}_\rho\right\rVert_2
\eq
where $\mathbf{w}_\rho$ is the shift used to generate $\config_\rho$
from $\config_{\rho-1}$ and the norm is from Eqn.~(\ref{shiftsize}).
This distance metric should not be taken too seriously since it
is not gauge invariant.
In Fig.~\ref{trajectory3}, we show some trajectories
in terms of the string tension $\Delta$ and $\sigma a^2$.

Lucini and Teper report a string tension $\sigma a^2=0.016282(76)$
for $N=3$, and $\beta=28$ (Ref.~\cite{lucini_$mathrmsun$_2002}, Table~5).
For the initial configuration $\config_0$ used in Fig.\ref{trajectory3},
we measure $\sigma a^2=0.0155(56)$;
with errors that remain about that size for $\config_k$, $k>0$.
For $\config_0$, $\plaquette_\mathrm{avg}=0.902$ and the
action is dominated by perturbative fluctuations of the gauge fields.
One can use the string tension to
estimate $\plaquette_\mathrm{avg}$ where we expect the action to become
dominated by the string-like behavior of the theory:
\be
\plaquette_\mathrm{avg} \gtrsim  1 - \sigma a^2 = 0.984 \; .
\eq


Ideally, $\plaquette_\mathrm{avg}$ will increase steadily and the string tension will remain relatively constant as we apply our iterative procedure.
This is illustrated by the trajectory {\bf 2d}.
In cases where $\gamma$ or $\Lambda_2$ is too large, we find that the
string tension does not remain stable, as illustrated by
trajectories {\bf 2b} and {\bf 2c}.

We also have investigated an ad-hoc procedure where
the off-diagonal color blocks of $H$ are dropped (that is, the
saddle point search is applied to each link individually).
This procedure is much easier to apply numerically.
As shown in trajectory {\bf lc},
we see that that the string tension can remain relatively stable,
but $\config_k$ eventually degrades to the trivial vacuum.

{\bf  To do:
  finish $\SU(3)$, $L^3=20^3$ calculation and
  add results for $\SU(4)$}


\section{Spectrum}

\begin{figure}
   configuration $\config_0$\\
   \includegraphics{corrOriginal}\\
   configuration $\config_{10}$\\
   \includegraphics{corr2d10}\\
   configuration $\config_{20}$\\
   \includegraphics{corr2d20Even}
\caption{Correlations of the even plaquette operator for several points
  along trajectory {\bf 2d} shown in Fig.~\ref{trajectory3}.
  Also shown is an exponential corresponding to Teper's value
  for the lowest $0^{++}$ glueball.
  \label{corr2dEven}}
\end{figure}

\begin{figure}
   \includegraphics{corr2d20Odd}
   \caption{Correlations of the odd plaquette operator for
     lattice configuration
     $\config_{20}$ on trajectory {\bf 2d} shown in Fig.~\ref{trajectory3}.
  Also shown is an exponential corresponding to Teper's value
  for the lowest $0^{--}$ glueball.
  \label{corr2dOdd}}
\end{figure}

\begin{figure}
   \includegraphics{corrLink995}
   \caption{Correlations of the even plaquette operator for
     the $\plaquette_\mathrm{avg} \approx 0.995$ lattice configuration
     on trajectory {\bf lc} shown in Fig.~\ref{trajectory3}.
    Also shown is an exponential corresponding to Teper's value
     for the lowest $0^{++}$ glueball.
     We do not see the desired exponential behavior.
  \label{corrLink995}}
\end{figure}

We can look at other properties of the gauge configuration
as we move along the trajectory.  In particular, let us look
at the correlators of the charge conjugation even plaquette operator
\be
  \left\langle
  \left(\Re \plaquette_{\mu,\nu}(x) - \plaquette_\mathrm{avg}\right)
  \left(\Re \plaquette_{\alpha,\beta}(y) - \plaquette_\mathrm{avg}\right)
  \right\rangle
\eq
as a function of the separation of the two plaquettes
\be
\mathrm{separation} =
        \left\lVert \left(x+a \hat{\mu}/2+ a\hat{\nu}/2\right) -
             \left(y+a \hat{\alpha}/2+ a\hat{\beta}/2\right)
             \right\rVert \; .
\eq
This correlator couples to the $0^{++}$ glueball states.

In Fig.~\ref{corr2dEven}, we plot the correlation for various relative
orientations of the two plaquettes for several points along the
best trajectory in~\ref{trajectory3}.  We also plot an exponential
corresponding to Teper's value for the $0^{++}$ glueball,
mass $0.5517(38)/a$ for $\beta=28$, $L^3=23^3$~\cite{teper_$mathrmsun$_1998}.
The exponential only demonstrates that there is qualitative agreement.
Since the correlator also has contributions from the excited $0^{++}$
states, a quantitative fit would require a more extensive analysis,
with additional operators in the correlation function.  However, one
would expect the single plaquette operator to couple very strongly
to the lowest state and not so strongly to the excited states.

Likewise, we can look at the charge conjugation odd correlator
\be
  \left\langle
  \left(\Im \plaquette_{\mu,\nu}(x) - \plaquette_\mathrm{avg}\right)
  \left(\Im \plaquette_{\alpha,\beta}(y) - \plaquette_\mathrm{avg}\right)
  \right\rangle   \; ,
\eq
corresponding to the $0^{--}$ glueball states; see Fig.~\ref{corr2dOdd}.
Also shown is an exponential corresponding to the lowest $0^{--}$ state,
mass $0.8133(57)/a$ \cite{teper_$mathrmsun$_1998}.

In the case where the off-diagonal color blocks of $H$ are dropped,
we do not see the desired exponential behavior in the correlator,
Fig.~\ref{corrLink995}.

\section{Gauge Fields}

{\bf To do:  look at the fields themselves.

  Ideas:
  \begin{itemize}
  \item  Look at $A_{\mu,a}$ directly in the $A_0=0$ (axial) gauge.

  \item Find some nice way to express the Faraday tensor $F_{\mu,\nu}$ and
    plot it.

  \item Impose fixed boundary conditions on a cylinder where we fix the
    link fields to correspond to one center charge inside the
    cylinder.
    
  \end{itemize}
}


\section{Conclusions}

The master field idea is a statement about the $N\to \infty$ limit
of $\SU(N)$ gauge theory.  However, if finite $N$ is similar
to the $N\to\infty$ limit, we should expect to properties of
the master field emerge, approximately, at finite $N$.
As an example of this, we
see that a saddle point of a single lattice configuration
contains the necessary physics, at least approximately, to produce the
low-energy spectrum of the theory.

\vspace{10mm}
\noindent {Acknowledgments}:

\appendix

\section{Roots of $\SU(N)$ matrices}
\label{roots}

In general, for matrix $U_c$ belonging to a connected Lie group,
the matrix power $U_c^t$, $t\in[0,1]$ should be a
smooth map from $[0,1]$ onto the group manifold where
$U_c^0=\mathbb{I}$, $U_c^1=U_c$, and $U_c^s U_c^t = U_c^{s+t}$;
the image of the map should have minimal length.
A unitary matrix $U$ can be factored as
%
\be
U = V^\da \begin{pmatrix}
    \mathrm{e}^{i \lambda_1} & & &\\
    & \mathrm{e}^{i \lambda_2} & &\\
    & & \ddots & \\
    & & & \mathrm{e}^{i \lambda_N}\end{pmatrix} V
\eq
%
where $V$ is a unitary matrix and $-\pi < \lambda_i < \pi$.
In the special case $U \in \SU(N)$, the $\det(U)=1$ condition implies that
\be
\sum_j \lambda_j = 2 \pi m\;, \quad m\in\integer \;.
\eq
When taking matrix roots, cases where $m\neq 0$ present a
difficulty.  However, one can always shift $\lambda_j$ by
multiples of $2\pi$:
\be
\lambda_j \to \lambda_j^\prime = \lambda_j + 2 \pi n_j\;,\quad
n_j\in\integer \; ,
\eq
such that $\sum_j \lambda_j^\prime = 0$ and
$\lambda_j^\prime - \lambda_k^\prime \le 2 \pi$.
%
\begin{figure}
\includegraphics{hex3}
\caption{In the case of $\SU(3)$, $(\lambda_1^\prime,\lambda_2^\prime)$
  lie in the shaded hexagonal region.
  Elements of the center of the group $z$, $z^2$, lie on the vertices
  of the hexagon with the identity at the origin.
  The six triangular regions correspond to permuations of
  $\lambda_1$, $\lambda_2$, $\lambda_3$.   \label{hexagon}}
\end{figure}
%
The $\SU(3)$ case is shown in Fig.~\ref{hexagon}.
Using $\lambda_j^\prime$, we define $U^t$ as:
\be
U^t = V^\da \begin{pmatrix}
    \mathrm{e}^{i\lambda_1^\prime t} & & &\\
    & \mathrm{e}^{i\lambda_2^\prime t} & &\\
    & & \ddots & \\
    & & & \mathrm{e}^{i\lambda_N^\prime t}\end{pmatrix} V \; .
\eq
This definition of $U^t$ fulfills the conditions listed above.
Similarly, we define the logarithm of $U$ as:
\be
\log U = i V^\da \begin{pmatrix}
    \lambda_1^\prime & & &\\
    & \lambda_2^\prime & &\\
    & & \ddots & \\
    & & & \lambda_N^\prime\end{pmatrix} V \; .
\eq
Thus, the norm defined in Eqn.~(\ref{sunorm}) is simply
\be
\left\lVert U \right\rVert_2 = \sqrt{\sum_j {\lambda_j^\prime}^2} \; .
\eq


\section{Solving a constrained linear system}
\label{krylov}

For lattices of interest, the basis size $n_L$ is large enough
that direct solution is not possible and one must resort to Krylov
space methods (sparse matrix methods).
The Hessian matrix $H$ is sparse, with an $(N^2-1)\times(N^2-1)$
color block structure and $6D+1$ nonzero blocks per row/column.
The gauge transform matrix $G$ has the same color block structure
with $2 D$ nonzero blocks per row and $2$ nonzero blocks per column.

In order to preserve the sparsity of the matrices, we never
explicitly construct the space of non-transform shifts used in
Eqn.~(\ref{linear2}).  Instead, we apply the projection
operator $P_G$ to any Krylov space vector $\mathbf{v}$ using:
\be
    P_G \mathbf{v} = \mathbf{v} - G^T \mathbf{u} \; ,
\eq
where $\mathbf{u}$ is a solution of the linear system
\be
  G G^T \mathbf{u} = G \mathbf{v} \; .\label{gcg}
\eq
Eqn.~(\ref{gcg}) is solved using the conjugate gradient
algorithm, subroutine MINRES-QLP~\cite{choi_algorithm_2014}.
The matrix $G G^T$ is well-conditioned, so this typically converges
in less than 40 steps.

We use the Lanczos library nu-TRLan~\cite{yamazaki_adaptive_2010},
modified so that $P_G$ is applied to any Krylov space vector, to
find the smallest eigenpairs of $\left(P_G H\right)^2$.  From these, we find
the eigenvectors $\left\{\mathbf{r}_1, \mathbf{r}_2, \ldots\right\}$
associated with eigenpairs $\left(\heigen_i^2, \mathbf{r}_i\right)$
that violate the $\Lambda$ cutoffs,
%
\begin{eqnarray}
   \frac{\sqrt{2}\, \mathbf{r}_i \mathbf{\cdot} \mathbf{q}
   \left\lVert \mathbf{r}_i\right\rVert_2}{\left|\heigen_i\right| }
     &>& \Lambda_2  \quad\mbox{or} \\
   \frac{\sqrt{2}\, \mathbf{r}_i\mathbf{\cdot} \mathbf{q}
   \left\lVert \mathbf{r}_i\right\rVert_\mathrm{max}}{\left|\heigen_i\right| }
     &>& \Lambda_\mathrm{max} \; .
\end{eqnarray}
%
To ensure that we have found all the eigenpairs that violate the cutoffs,
we may need to calculate as many as 2000 to 4000 of the lowest eigenpairs
in the Lanzos routine.  As one moves along a trajectory towards the
saddle point, the gradient vector decreases in magnitude and fewer eigenpairs
exceed the $\Lambda$ cutoffs.

Then we solve the linear system $H \mathbf{y} = \mathbf{q}$
using the conjugate gradient algorithm,
subroutine MINRES-QLP~\cite{choi_algorithm_2014}, modified so that
the projection operators $P_G$ and 
$\mathbb{I}- \sum_{i} \mathbf{r}_i \otimes \mathbf{r}_i$
are applied to any Krylov space vector.

\section{Lattice configurations}
\label{configurations}

Lattice configurations were generated using the Chroma
code~\cite{edwards_chroma_2005}.
For $N>3$ it was necessary to add partial pivoting~\cite{golub_matrix_1996}
to the Chroma routine that calculates matrix determinants.  We then
verified that Chroma-generated values of $\plaquette_\mathrm{avg}$ agree with
Teper's results~\cite{lucini_$mathrmsun$_2002} for $N=3,4$.

The lattice configurations were sampled after 10,000 thermalizing
steps, with 3000 steps between samples for results that use
multiple lattice configurations.  We used 4 over-relaxations steps
for every heat bath step.

\bibliography{physics}  % Use physics.bib

\end{document}
