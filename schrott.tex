%\documentclass[twocolumn,eqsecnum,aps,]{revtex4-2}
\documentclass[preprint,aps,prd]{revtex4-2}
%\documentclass[eqsecnum,aps]{revtex}
%
%
%  Using Zotero to generate physics.bib file.
%
%  A number of sites set language=en which Zotero
%  passes on to BibTex, but LaTeX package babel doesn't know ``en.''
%  In Zotero, either remove the language field or set it to ``english''
%  before exporting to BibTex.
%\usepackage[english]{babel}
%
%
%  Ubuntu doesn't have the latest RevTex.
%  Download and install using:
%    sudo unzip revtex4-2-tds.zip -d /usr/share/texlive/texmf-dist/
%    sudo mktexlsr /usr/share/texlive/texmf-dist/
%
%  RevTex 4.2 does not properly handle very long author lists.
%  Run this to truncate the author list for the 2014 Brambilla article:
%    perl -pi -e 's/and Eidelman[^}]*/and others/g' physics.bib
%
\bibliographystyle{apsrev4-2}
%
\usepackage{amsmath}
\usepackage{mathtools} % For MoveEqnLeft
\usepackage{amsfonts}
\usepackage{graphicx}
\usepackage{afterpage} % For figure placement
%
\newcommand{\da}{\dagger}  % symbol for Hermitian conjugate dagger
\newcommand{\be}{\begin{equation}}
\newcommand{\eq}{\end{equation}}
\newcommand{\integer}{\mathbb{Z}}       % set of integers
\newcommand{\zentrum}{\mathcal{Z}}       % set of integers
\newcommand{\plaquette}{\Box}
\newcommand{\config}{\mathcal{U}}
\newcommand{\orelax}{\xi}
\newcommand{\heigen}{h}
\newcommand\wilson[4]{\Omega_{#1, #2}\left(#3,#4\right)}
\DeclareMathOperator{\SU}{SU}
\DeclareMathOperator{\Tr}{Tr}
\newcommand\fnorm[1]{\left\lVert #1 \right\rVert_\mathrm{F}}
\newcommand\znorm[1]{\left\lVert #1 \right\rVert_\zentrum}
\newcommand\cov[2]{\mathrm{cov}\!\left(#1, #2\right)}

\begin{document}

\section{Operator distributions}

\begin{figure}
  \includegraphics{polydist1}
  \includegraphics{polydistr}
  \caption{Distribution for the Wilson loop operator
    $\wilson{\mu}{\nu}{5}{5}$ for $\SU(3)$, $\beta=28$, $L^3=16^3$,
    averaging over 100 lattice configurations.
    Also shown is the distribution for the strong coupling limit.
    The distribution is
    symmetric under charge conjugation,
    % $2 \lambda_1 +\lambda_2 \to 2 \lambda_1 +\lambda_2 $,
    $\lambda_2 \to -\lambda_2$.
   \label{polydist1}}
\end{figure}
% \begin{figure}
%  \includegraphics{polydist2}
%  \caption{Distribution for $\Tr \wilson{\mu,\nu}{5}{5}/N$  for
%    the data shown in Fig.~\ref{polydist1}.  
%   The distribution is symmetric under complex conjugation,
%    $\mathcal{O}\to\mathcal{O}^\dagger$.
%    The distribution lies inside the blue lines representing
%    the edges of the triangle in Fig.~\ref{polydist1}.
%    \label{polydist2}}
%\end{figure}

The quantity that is actually measured can be generalized, as well.
For some loop operator $\mathcal{O}\in\SU(N)$, one usually
measures the expectation value of the trace,
$\langle 0 | \Tr \mathcal{O} |0\rangle$, or for $k$-strings,
one measures expectation values of higher order operators
like $\Tr\left( \mathcal{O}^k\right)$.
However, this is not the most general gauge-invariant observable.
The most general physical observable is the distribution of the
eigenvalues of $\mathcal{O}$.  If $e^{i\lambda_1}$,\ldots, $e^{i\lambda_N}$
are the eigenvalues of $\mathcal{O}$, we can order $\lambda_j$ in
some fashion $\lambda_1\ge \lambda_2 \ge \ldots$ and demand that
$\lambda_j-\lambda_k\le 2\pi$ (see Appendix~\ref{roots}).
In that case, physically
distinct values lie on an ($N-1$)-simplex with the elements of the
center of $\SU(N)$ at the vertices.

By way of example, we show the distribution of the
$5\times5$ Wilson loop operator for $\SU(3)$ in Fig.~\ref{polydist1}.
For comparison, we also show the distribution for
the strong coupling limit (random $\SU(N)$ matrices).
Notice that the shape of the two distributions appear to
be quite similar, except that the $\wilson{\mu}{\nu}{5}{5}$ distribution
is shifted toward $\mathbb{I}$, reflecting the fact that
$\langle\Tr \wilson{\mu}{\nu}{5}{5}\rangle >0$.

% We can relate the distribution in $\lambda_j$-space to the
% distribution of $\Tr\mathcal{O}$.
% For $\SU(2)$, $\Tr \mathcal{O}$ is real and for $\SU(3)$, $\Tr \mathcal{O}$
% is complex.  In both cases, the map from the $\lambda_j$-space to
% $\Tr \mathcal{O}$ is injective and one can just as well look at the
% distribution of $\Tr \mathcal{O}$; see Fig.~\ref{polydist2}.
% However, for $N>3$, information is lost when taking the trace
% of $\mathcal{O}$.


\section{Coulomb corrections}
\label{coulomb}

In $D=4$ spacetime directions, there is a clear separation---both
experimentally and in lattice calculations---between a
short distance regime where the heavy quark potential is
dominated by a $1/r$ Coulomb force potential and a
long-distance regime having a linear confining
potential~\cite{bali_static_2000,brambilla_effective-field_2005}.
In $D=3$, the Coulomb potential goes as $\log(r)$ and distinguishing
it from the linear confining potential is more difficult.
In this section, we derive
the functional form of the Coulomb contribution to
the effective potential for Polyakov loop correlators and Wilson
loops.

If we consider a pair of charges separated by some
distance $r$ at some instant in time, the Coulomb potential is
equal to the energy stored in the electric
field produced by those charges (electrostatics in $D-1$
dimensions).  Equivalently, we can consider currents
in $D$ Euclidean spacetime dimensions.  In that case,
the classical action is equal to the
energy of the magnetic field produced by those currents
(magnetostatics in $D$ dimensions).  Using the textbook
formula~\cite{jackson_classical_1975}
for the magnetostatic energy associated with a current density
$\mathbf{J}(\mathbf{x})$,
%
\be
S = \frac{1}{2} \int d^3\mathbf{x}\, d^3\mathbf{x}^\prime\,
\frac{\mathbf{J}(\mathbf{x}) \cdot
  \mathbf{J}(\mathbf{x}^\prime)}{
  \left| \mathbf{x}-\mathbf{x}^\prime\right|} \, ,
\eq
%
we can find the action associated with a number of thin wires
carrying a constant current. (We set the current equal to 1 in
the following.)

In the case of two Polyakov loops with opposite currents separated
by distance $r$ with periodic boundary conditions, period $L$,
there is a self-energy term plus an interaction term:
%
\begin{eqnarray}
S_\Phi &=& 2 \int_0^L dx \int_{x+\varepsilon/2}^\infty dy
\left(\frac{1}{y-x} -\frac{1}{\sqrt{(y-x)^2 + r^2}}\right) \\
  &=& 2 L \log\left(\frac{r}{\varepsilon}\right) + O(\varepsilon)\, .
\end{eqnarray}
%
Thus, we obtain the usual $L \log(r)$ term plus a
self-energy term proportional to the length
of the currents, $2L$.  The $\log(\varepsilon)$ divergence
is from approximating the current density as a thin wire.
Since the normalization
of this field relative to the bare link fields is
unknown, one can fit the Coulomb contribution to the
Polyakov loop correlator using
\be
       c_q 2 L \log(r) + c_p 2 L \, , \label{cpf}
\eq
where $c_q$ and $c_p$ are unknown.

In the case of a $w_1\times w_2$ Wilson loop, we obtain
a more complicated expression
%
\begin{eqnarray}
  S_W &=& 2 \int_0^{w_1} dx \int_{x+\varepsilon}^{w_1} \frac{dx^\prime}{x^\prime-x}
               \nonumber\\
               & &  - \int_0^{w_1} dx \int_{0}^{w_1} \frac{dx^\prime}{
                 \sqrt{(x^\prime-x)^2+ w_2^2}}
               \nonumber\\
               & &  +\left[w_1 \leftrightarrow w_2 \right]\\
               &=&  \Theta\!\left(w_1, w_2\right) +
               \Theta\!\left(w_2, w_1\right)
              - 2 (w_1 + w_2) \log(\varepsilon) + O(\varepsilon) 
\end{eqnarray}
where
\be
\Theta\!\left(\ell, r\right) = 2 \left(
    \sqrt{\ell^2+r^2} - \ell - r
    - \ell\log\left(\frac{\sqrt{\ell^2+r^2} + \ell}{r\ell}\right)
   \right) \;.
\eq
Similar the the above, 
we obtain a self-energy term that is proportional to the perimeter of
the rectangle.  Thus, we can fit the Coulomb
contribution to the Wilson loop using
%
\be
    c_q  \left(\Theta\!\left(w_1, w_2\right) +
               \Theta\!\left(w_2, w_1\right)\right)
        + c_p  2\left(w_1+w_2\right)  \label{cwf}
\eq
%
where $c_p$ and $c_q$ in Eqns.~(\ref{cpf}) and (\ref{cwf})
should be equivalent.


\bibliography{physics}  % Use physics.bib

\end{document}
